\documentclass{article}
\usepackage[utf8]{inputenc}
\usepackage[english]{babel}
\usepackage[margin=0.8in]{geometry}
\usepackage[symbol]{footmisc}
\usepackage[titles]{tocloft}
\usepackage[shortlabels]{enumitem}

\usepackage{amsmath, amssymb, MnSymbol, wasysym, varwidth, enumitem, array, caption, floatrow, wasysym, tikz, tikz-cd, graphicx, hyperref, dsfont, tabto}

\renewcommand{\cftsecfont}{\normalfont\mdseries}
\usepackage{hyperref}
\hypersetup{
    colorlinks,
    citecolor=blue,
    filecolor=blue,
    linkcolor=blue,
    urlcolor=blue
}

\newcommand{\R}{\mathbb{R}}
\newcommand{\C}{\mathbb{C}}
\newcommand{\Z}{\mathbb{Z}}
\newcommand{\Q}{\mathbb{Q}}
\newcommand{\I}{[0, 1]}
\newcommand{\h}{\simeq}
\newcommand{\identity}{\mathds{1}}
\newcommand{\Hom}{\textnormal{Hom}}
\newcommand{\Ext}{\textnormal{Ext}}
\newcommand{\Tor}{\textnormal{Tor}}
\newcommand{\Char}{\textnormal{Char}}
\newcommand{\im}{\textnormal{im}}

\setcounter{section}{-1}

\title{Hatcher's Algebraic Topology - Solutions}
\author{Institute for Pure and Applied Mathematics (IMPA)}
\date{Written by: Tiam Koukpari}

\begin{document}
\maketitle

Trying to collect the fragmented sets of solutions into one file. Here is the sequence of requisites needed for this topic:
\bigskip

\[\begin{tikzcd}
& \textnormal{Differential Forms (de Rham Cohomology)} \arrow[d] & \\
& \textnormal{Analysis in } \R^{n} \subseteq \textnormal{Topology} \subseteq \textnormal{Manifolds} \arrow[rrd] \arrow[u] & \\
\textnormal{Set Theory} \arrow[ru] \arrow[rd] & & & \textnormal{Algebraic Topology} \\
& \textnormal{Groups} \subseteq \textnormal{Rings} \subseteq \textnormal{Fields} \arrow[r]\arrow[d] & \textnormal{Linear Algebra} \arrow[ru] & \\
& \textnormal{Homological Algebra \& Categories} \arrow[u]
\end{tikzcd}\]
\bigskip
\bigskip

References, if used, are included at the end of each exercise.
\medskip

If you find any mistakes or if you want to submit a solution, please email \href{mailto:tiam.koukpari@impa.br}{tiam.koukpari@impa.br}. The remaining problems are:
\medskip

\textbf{Chapter 0}:
\medskip

3, 4, 5, 6, 7, 8, 9, 10, 11, 12, 13, 14, 15, 16, 17, 18, 19, 20, 21, 22, 23, 24, 25, 26, 27, 28, 29
\medskip

\textbf{Chapter 1}:
\medskip

1.1, 1.2, 1.3, 1.4, 1.5, 1.6, 1.7, 1.8, 1.9, 1.10, 1.11, 1.12, 1.13, 1.14, 1.15, 1.16, 1.17, 1.18, 1.19, 1.20

2.1, 2.2, 2.3, 2.4, 2.5, 2.6, 2.7, 2.8, 2.9, 2.10, 2.11, 2.12, 2.13, 2.14, 2.15, 2.16, 2.17, 2.18, 2.19, 2.20, 2.21, 2.22

3.1, 3.2, 3.3, 3.4, 3.5, 3.6, 3.7, 3.8, 3.9, 3.10, 3.11, 3.12, 3.13, 3.14, 3.15, 3.16, 3.17, 3.18, 3.19, 3.20, 3.21, 3.22, 3.23, 3.24, 3.25, 3.26, 3.27, 3.28, 3.29, 3.30, 3.31, 3.32, 3.33

A.1, A.2, A.3, A.4, A.5, A.6, A.7, A.8, A.9, A.10, A.11, A.12, A.13, A.14

B.1, B.2, B.3, B.4, B.5, B.6, B.7, B.8, B.9
\medskip

\textbf{Chapter 2}:
\medskip

1.2, 1.3, 1.6, 1.7, 1.8, 1.10, 1.12, 1.13, 1.14, 1.17, 1.18, 1.20, 1.21, 1.22, 1.23, 1.24, 1.26, 1.27, 1.28, 1.29, 1.30

2.2, 2.5, 2.7, 2.8, 2.9, 2.10, 2.11, 2.12, 2.13, 2.14, 2.15, 2.16, 2.17, 2.18, 2.19, 2.22 2.23, 2.24, 2.25, 2.26, 2.28, 2.29, 2.30, 2.31, 2.33, 2.35, 2.36, 2.38, 2.39, 2.40, 2.42, 2.43

3.1, 3.2, 3.3, 3.4

B.1, B.2, B.3, B.5, B.6, B.7, B.8, B.9, B.10, B.11

C.1, C.2, C.3, C.4, C.5, C.6, C.7, C.8, C.9
\medskip

\textbf{Chapter 3}:
\medskip

1.1, 1.2, 1.3, 1.4, 1.5, 1.6, 1.7, 1.8, 1.11, 1.12, 1.13

2.1, 2.2, 2.3, 2.4, 2.5, 2.6, 2.7, 2.8, 2.9, 2.10, 2.11, 2.12, 2.13, 2.14, 2.15, 2.16, 2.17, 2.18

3.1, 3.2, 3.3, 3.4, 3.5, 3.6, 3.7, 3.8, 3.9, 3.10, 3.11, 3.12, 3.13, 3.14, 3.15, 3.17, 3.18, 3.19, 3.20, 3.21, 3.22, 3.23, 3.24, 3.25, 3.26, 3.27, 3.28, 3.29, 3.30, 3.31, 3.32, 3.33, 3.34, 3.35

A.1, A.2, A.3, A.4, A.5, A.6

B.1, B.2, B.3, B.4, B.5

C.1, C.2, C.3, C.4, C.5, C.6, C.7, C.8, C.9, C.10, C.11, C.12, C.13, C.14, C.15, C.16

D.1, D.2, D.3

E.1, E.2, E.3, E.4

F.1, F.2, F.3, F.4, F.5, F.6, F.7, F.8, F.9

H.1, H.2, H.3, H.4, H.5, H.6
\medskip

\textbf{Chapter 4}:
\medskip

1.2, 1.3, 1.4, 1.5, 1.6, 1.7, 1.8, 1.9, 1.10, 1.11, 1.12, 1.13, 1.14, 1.15, 1.16, 1.17, 1.18, 1.19, 1.20, 1.21, 1.22, 1.23

2.1, 2.2, 2.3, 2.4, 2.5, 2.6, 2.7, 2.8, 2.9, 2.10, 2.11, 2.12, 2.13, 2.14, 2.15, 2.16, 2.17, 2.18, 2.19, 2.20, 2.21, 2.22 2.23, 2.24, 2.25, 2.26, 2.27, 2.28, 2.29, 2.30, 2.31, 2.32, 2.33, 2.34, 2.35, 2.36, 2.37, 2.38, 2.39

3.1, 3.2, 3.3, 3.4, 3.5, 3.6, 3.7, 3.8, 3.9, 3.10, 3.11, 3.12, 3.13, 3.14, 3.15, 3.16, 3.17, 3.18, 3.19, 3.20, 3.21, 3.22, 3.23, 3.24

A.1, A.2, A.3, A.4, A.5

B.1, B.2

D.1, D.2, D.3, D.4, D.5, D.6, D.7, D.8, D.9, D.10

F.1, F.2, F.3

G.1, G.2, G.3, G.4

H.1, H.2, H.3, H.4

I.1 I.2, I.3

J.1

K.1, K.2, K.3, K.4, K.5, K.6

L.1, L.2, L.3, L.4, L.5
\newpage

\tableofcontents
\newpage

\section{Some Underlying Geometric Notions}

\tab\textbf{1}. Construct an explicit deformation retraction of the torus with one point deleted onto a graph consisting of two circles intersecting in a point, namely, longitude and meridian circles of the torus.
\medskip

\textbf{Solution}. It is useful to visualize the torus with a $2\times 2$ square centered at the origin:
\usetikzlibrary{decorations.markings,positioning}

\[\begin{tikzpicture}[node distance=2cm]
\coordinate (top-right) {};
\coordinate[label=center:$\circ$]  (origin) at (-1,-1);
\coordinate[left=of top-right] (top-left) {};
\coordinate[below=of top-right] (bottom-left) {};
\coordinate[below=of top-left] (bottom-right) {};
\begin{scope}[decoration={markings, mark=at position 0.5 with {\arrow{<}}}]
\draw[postaction={decorate}] (top-right) -- node[auto,swap] {$v$} (top-left);
\draw[postaction={decorate}] (bottom-right) -- node[auto] {$w$} (top-left);
\end{scope}
\begin{scope}[decoration={markings, mark=at position 0.5 with {\arrow{>}}}]
\draw[postaction={decorate}] (bottom-right) -- node[auto,swap] {$v$} (bottom-left);
\draw[postaction={decorate}] (top-right) -- node[auto] {$w$} (bottom-left);
\end{scope}
\end{tikzpicture}\]

To form a torus, fold the shape to connect $v$ with itself, creating two copies of $S^{1}$ on $w$. Then fold the shape to connect $w$ with itself, joining the two copies of $S^{1}$ on $w$ and creating another $S^{1}$ on $v$. Without loss of generality, assume the deleted point is at the origin. As in the reference, construct

$$f_{t}(x, y) = (1-t)(x, y) + t\left(\frac{(x, y)}{\max\{|x|, |y|\}}\right).$$

Then $f_{0}(x, y) = (x, y)$ so that $f_{0} = \identity$, $f_{1}(x, y) = (x, y) / \max\{|x|, |y|\}$ so that $f_{1} = S^{1}\vee S^{1}$, and $f_{1}|S^{1}\vee S^{1} = \identity$ since $\max \{|x|, |y|\} = 1$ on the boundary. The function is continuous since $(0, 0)$ is not in its domain. $\blacksquare$
\medskip

\textit{Remark}. This may or may not be an acceptable solution, depending on whether `explicit' means `3-space.'
\medskip

References: \href{https://riemannianhunger.wordpress.com/solutions-to-algebraic-topology-by-allen-hatcher/hatcher-0-1/}{1}.
\bigskip
\bigskip

\textbf{2}. Construct an explicit deformation retraction of $\R^{n} - \{0\}$ onto $S^{n-1}$.
\medskip

\textbf{Solution}. Construct

$$f_{t}(\mathbf{x}) = (1-t)\mathbf{x} + t\frac{\mathbf{x}}{|\mathbf{x}|}.$$

Then $f_{0}(\mathbf{x}) = \mathbf{x}$ so that $f_{0} = \identity$, $f_{1}(\mathbf{x}) = \mathbf{x}/|\mathbf{x}|$ so that $f_{1} = S^{n-1}$, and $f_{t}|S^{n-1} = \identity$. The function is a straight, continuous line from $\mathbf{x}$ to a normalized $\mathbf{x}$, i.e. on the $(n-1)$-sphere. The function is continuous since $\{0\}$ is not in its domain. $\blacksquare$
\bigskip
\bigskip

\textbf{3}. (a) Show that the composition of homotopy equivalence $X\to Y$ and $Y\to Z$ is a homotopy equivalence $X\to Z$. Deduce that homotopy equivalence is an equivalence relation.
\medskip

\textbf{Solution}. Let $f:X\to Y$ be a homotopy equivalence and $f^{-1}:Y\to X$ its inverse. Similarly, let $g:Y\to Z$ be a homotopy equivalence and $g^{-1}:Z\to Y$ its inverse. Construct $h := g\circ f$ and $h^{-1} := f^{-1}\circ g^{-1}$. We want to show that $h\circ h^{-1}\simeq \identity$:

$$h\circ h^{-1} = g\circ f\circ f^{-1}\circ g^{-1}\simeq g\circ \identity\circ g^{-1} = g\circ g^{-1}\simeq \identity.$$

(b) Show that the relation of homotopy among maps $X\to Y$ is an equivalence relation.
\medskip

\textbf{Solution}. $\square$
\medskip

(c) Show that a map homotopic to a homotopy equivalence is a homotopy equivalence.
\medskip

\textbf{Solution}. $\blacksquare$
\bigskip
\bigskip

\textbf{4}. A \textbf{deformation retraction in the weak sense} of a space $X$ to a subspace $A$ is a homotopy $f_{t}: X\to X$ such that $f_{0} = \identity$, $f_{1}(X)\subset A$, and $f_{t}(A)\subset A$ for all $t$. Show that if $X$ deformation retracts to $A$ in this weak sense, then the inclusion $A\hookrightarrow X$ is a homotopy equivalence.
\medskip

\textbf{Solution}. $\blacksquare$
\bigskip
\bigskip

\textbf{5}. Show that if a space $X$ deformation retracts to a point $x\in X$, then for each neighborhood $U$ of $x$ in $X$ there exists a neighborhood $V\subset U$ of $x$ such that the inclusion map $V\hookrightarrow U$ is nullhomotopic.
\medskip

\textbf{Solution}. $\blacksquare$
\newpage

\section{The Fundamental Group}

\subsection{Basic Constructions}

\subsection{Van Kampen's Theorem}

\subsection{Covering Spaces}

\subsection*{Additional Topics}
\addcontentsline{toc}{section}{\protect\numberline{}Additional Topics}

\subsubsection*{1.A. Graphs and Free Groups}
\addcontentsline{toc}{subsection}{\protect\numberline{}1.A. Graphs and Free Groups}

\subsubsection*{1.B. K(G,1) Spaces and Graphs of Groups}
\addcontentsline{toc}{subsection}{\protect\numberline{}1.B. K(G,1) Spaces and Graphs of Groups}

\newpage

\section{Homology}

\subsection{Simplicial and Singular Homology}

\tab\textbf{1}. What familiar space is the quotient $\Delta$-complex of a 2 simplex $[v_{0}, v_{1}, v_{2}]$ obtained by identifying the edges $[v_{0}, v_{1}]$ and $[v_{1}, v_{2}]$, preserving the ordering of vertices?
\medskip

\textbf{Solution} The Möbius strip. We draw the same construction as in the reference:

\[\begin{tikzpicture}[scale=0.5]
\coordinate[label=left:$v_{0}$]  (v0) at (0,0);
\coordinate[label=right:$v_{1}$] (v1) at (4,0);
\coordinate[label=above:$v_{2}$] (v2) at (2,3.464);
\coordinate (1/2) at (1,1.732);
\draw (v0) -- (v1) -- (v2) -- cycle;
\begin{scope}[decoration={markings, mark=at position 0.5 with {\arrow{>}}}]
\draw[postaction={decorate}, color=red] (v0) -- node[auto] {} (v1);
\draw[postaction={decorate}] (v0) -- node[auto] {} (v2);
\draw[postaction={decorate}, color=red] (v1) -- node[auto] {} (v2);
\end{scope}
\begin{scope}
\draw[dashed, color=blue] (v1) -- node[auto] {} (1/2);
\end{scope}
\end{tikzpicture}\quad\quad
\begin{tikzpicture}[scale=0.5]
\coordinate[label=left:$v_{0}$]  (v0) at (0,0);
\coordinate[label=right:$v_{1}$] (v1) at (4,0);
\coordinate (v11) at (4.5,0.5);
\coordinate[label=above:$v_{2}$] (v2) at (2.5,3.964);
\coordinate[label=above:$\hat{v}$] (1/2) at (1,1.732);
\coordinate (1/21) at (1.5,2.232);
\draw (v0) -- (v1) -- (1/2) -- cycle;
\draw (v11) -- (v2) -- (1/21) -- cycle;
\begin{scope}[decoration={markings, mark=at position 0.5 with {\arrow{>}}}]
\draw[postaction={decorate}, color=red] (v0) -- node[auto] {} (v1);
\draw[postaction={decorate}] (v0) -- node[auto] {} (1/2);
\draw[postaction={decorate}, color=blue] (v1) -- node[auto] {} (1/2);
\draw[postaction={decorate}, color=red] (v11) -- node[auto] {} (v2);
\draw[postaction={decorate}] (1/21) -- node[auto] {} (v2);
\draw[postaction={decorate}, color=blue] (v11) -- node[auto] {} (1/21);
\end{scope}
\end{tikzpicture}\quad \quad
\begin{tikzpicture}[scale=0.5]
\coordinate[label=left:$v_{0}$]  (v0) at (0,0);
\coordinate[label=right:$v_{1}$] (v1) at (4,0);
\coordinate[label=left:$v_{1}$] (v11) at (0,-0.5);
\coordinate[label=right:$v_{2}$] (v2) at (4,-0.5);
\coordinate[label=above:$\hat{v}$] (1/2) at (2,1.732);
\coordinate[label=below:$\hat{v}$] (1/21) at (2,-2.232);
\draw (v0) -- (v1) -- (1/2) -- cycle;
\draw (v11) -- (v2) -- (1/21) -- cycle;
\begin{scope}[decoration={markings, mark=at position 0.5 with {\arrow{>}}}]
\draw[postaction={decorate}, color=red] (v0) -- node[auto] {} (v1);
\draw[postaction={decorate}] (v0) -- node[auto] {} (1/2);
\draw[postaction={decorate}, color=blue] (v1) -- node[auto] {} (1/2);
\draw[postaction={decorate}, color=red] (v11) -- node[auto] {} (v2);
\draw[postaction={decorate}] (1/21) -- node[auto] {} (v2);
\draw[postaction={decorate}, color=blue] (v11) -- node[auto] {} (1/21);
\end{scope}
\end{tikzpicture}\quad\quad
\begin{tikzpicture}[scale=0.5]
\coordinate[label=left:$v'$]  (v1) at (0,0);
\coordinate[label=right:$\hat{v}$] (v2) at (4,0);
\coordinate[label=above:$v'$] (v3) at (4,4);
\coordinate[label=above:$\hat{v}$] (v4) at (0,4);
\draw (v1) -- (v2) -- (v3) -- (v4) -- cycle;
\begin{scope}[decoration={markings, mark=at position 0.5 with {\arrow{>>}}}]
\draw[postaction={decorate}, color=blue] (v1) -- node[auto] {} (v2);
\draw[postaction={decorate}, color=blue] (v3) -- node[auto] {} (v4);
\end{scope}
\begin{scope}[decoration={markings, mark=at position 0.5 with {\arrow{>}}}]
\draw[postaction={decorate}] (v1) -- node[auto] {} (v4);
\draw[postaction={decorate}] (v2) -- node[auto] {} (v3);
\end{scope}
\end{tikzpicture}\]

The latter being the Möbius strip. $\blacksquare$
\medskip

References: \href{https://riemannianhunger.wordpress.com/solutions-to-algebraic-topology-by-allen-hatcher/hatcher-2-1-1/}{1}.
\bigskip
\bigskip

\textbf{2}. Show that the $\Delta$-complex obtained from $\Delta^{3}$ by performing the order-preserving edge identifications $[v_{0}, v_{1}]\sim [v_{1}, v_{3}]$ and $[v_{0}, v_{2}]\sim [v_{2}, v_{3}]$ deformation retracts onto a Klein bottle. Also, find the other pairs of identifications of edges that produce $\Delta$-complexes deformation retracting onto a torus, a 2-sphere, and $\R P^{2}$.
\medskip

\textbf{Solution}. $\blacksquare$
\bigskip
\bigskip

\textbf{3}. Construct a $\Delta$-complex structure on $\R P^{n}$ as a quotient of a $\Delta$-complex structure on $S^{n}$ having vertices the two vectors of length along each coordinate axis in $\R^{n+1}$.
\medskip

\textbf{Solution}. $\blacksquare$
\bigskip
\bigskip

\textbf{4}. Compute the simplicial homology groups of the triangular parachute obtained from $\Delta^{2}$ by identifying its three vertices to a single point.
\medskip

\textbf{Solution}. The face is generated by $U$, edges by $a, b$ and $c$, and vertex by $v$. The boundary operators (according to the ordering given in the reference) are given by

$$\partial U_{2} = b - c + a,\quad \partial_{1} a = \partial_{1} b = \partial_{1} c = \partial_{0} v = 0.$$

The first two homology groups are then given by

$$H^{\Delta}_{0} = \frac{\ker (\partial_{0})}{\im(\partial_{1})} = \frac{\langle v\rangle}{0} = \Z,\quad H^{\Delta}_{1} = \frac{\ker (\partial_{1})}{\im(\partial_{2})} = \frac{\langle a, b, c\rangle}{\langle b-c+a\rangle} = \frac{\langle a, b, b-c+a\rangle}{\langle b-c+a\rangle} = \langle a, b\rangle = \Z^{2}.$$

For $k\geq 2$, $H_{k}^{\Delta} = 0$. $\blacksquare$
\medskip

References: \href{http://web.math.ku.dk/~moller/blok1_05/AT-ex.pdf}{1}.
\bigskip
\bigskip

\textbf{5}. Compute the simplicial homology groups of the Klein bottle using the $\Delta$-complex structure described at the beginning of this section.
\medskip

\textbf{Solution}. The faces are generated by $U$ and $L$, edges by $a, b$ and $c$, and vertex by $v$. The boundary operators are given by

$$\partial_{2} U = a + b - c,\quad \partial_{2} L = a - b + c,\quad \partial_{1} a = \partial_{1} b = \partial_{1} c = \partial_{0} v = 0.$$

The first two homology groups are then given by

$$H^{\Delta}_{0} = \frac{\ker (\partial_{0})}{\im(\partial_{1})} = \frac{\langle v\rangle}{0} = \Z,\quad H^{\Delta}_{1} = \frac{\ker (\partial_{1})}{\im(\partial_{2})} = \frac{\langle a, b, c\rangle}{\langle a+b-c, a-b+c\rangle} = \frac{\langle a, b, c\rangle}{\langle a + b - (b - a), c\rangle} = \frac{\langle a, b\rangle}{\langle 2a\rangle} = \Z_{2}\oplus \Z.$$

For $k\geq 2$, $H_{k}^{\Delta} = 0$. $\blacksquare$
\bigskip
\bigskip

\textbf{6}. Compute the simplicial homology groups of the $\Delta$-complex obtained from $n+1$ 2-simplices $\Delta_{0}^{2},\ldots, \Delta_{n}^{2}$ by identifying all three edges of $\Delta_{0}^{2}$ to a single edge, and for $i > 0$ identifying the edges $[v_{0}, v_{1}]$ and $[v_{1}, v_{2}]$ of $\Delta_{i}^{2}$ to a single edge and the edge $[v_{0}, v_{2}]$ to the edge $[v_{0}, v_{1}]$ of $\Delta_{i-1}^{2}$.
\medskip

\textbf{Solution}. $\blacksquare$
\bigskip
\bigskip

\textbf{7}. Find a way of identifying pairs of faces of $\Delta^{3}$ to produce a $\Delta$-complex structure on $S^{3}$ having a single 3-simplex, and compute the simplicial homology groups of this $\Delta$-complex.
\medskip

\textbf{Solution}. Identify any two faces together, then identify the remaining two faces together. For example, identify 123 with 023, and 012 with 013. The construction is a single 3-simplex 0123, two 2-simplices $123 = 023$ and $012 = 013$, three 1-simplices $02 = 03 = 12 = 13$, 01 and 23, and two 0-simplices $0 = 1$ and $2 = 3$.
\medskip

We can easily compute $\partial_{3}(0123) = 0$. Next we compute the lower-dimensional boundaries:

$$\partial_{2} (123) = \partial_{2} (023) = (23)-(03)+(02) = (23),\quad \partial_{2} (012) = \partial_{2} (013) = (13)-(03)+(01) = (01)$$
$$\partial_{1}(02) = \partial_{1}(03) = \partial_{1}(12) = \partial_{1}(13) = (1) - (3),\quad \partial_{1}(01) = 0,\quad \partial_{1}(23) = 0$$
$$\partial_{0}(0) = \partial_{0}(1) = 0,\quad \partial_{0}(2) = \partial_{0}(3) = 0.$$

The simplicial homology groups are then

$$H^{\Delta}_{0} = \frac{\ker (\partial_{0})}{\im(\partial_{1})} = \frac{\langle (0), (2)\rangle}{\langle (1)-(3), 0, 0\rangle} = \Z,\quad H^{\Delta}_{1} = \frac{\ker (\partial_{1})}{\im(\partial_{2})} = 0 = \frac{\ker (\partial_{2})}{\im(\partial_{3})} = H^{\Delta}_{2}$$
$$\quad H^{\Delta}_{3} = \frac{\ker (\partial_{3})}{\im(\partial_{4})} = \frac{\langle (0123)\rangle}{0} = \Z.\quad \blacksquare$$
\bigskip

\textbf{9}. Compute the homology groups of the $\Delta$-complex $X$ obtained from $\Delta^{n}$ by identifying all faces of the same dimension. Thus $X$ has a single $k$ simplex for each $k\leq n$.
\medskip

\textbf{Solution}. The boundary operators are given by $\partial_{k} = \sum_{i=0}^{k} (-1)^{i} [v_{1},\ldots, \hat{v}_{i},\ldots, v_{k}] = 0$ if $k$ is even and $[v_{1},\ldots, v_{k-1}]$ otherwise. If $k > n$, then $H^{\Delta}_{k} = 0$. If $k < n$, then

$$H^{\Delta}_{k} = \frac{\ker (\partial_{k})}{\im(\partial_{k+1})} = \begin{cases}\Z/\Z = 0 & \textnormal{if $k$ is even}\\ 0/0 = 0 & \textnormal{if $k$ is odd}.\end{cases}$$

Finally, since the image of $\partial_{n+1}$ is 0, if $k = n$, then

$$H^{\Delta}_{n} = \ker (\partial_{n}) = \begin{cases}\Z & \textnormal{if $n$ is even}\\ 0 & \textnormal{if $n$ is odd}.\end{cases}\quad \blacksquare$$
\bigskip

\textbf{11}. Show that if $A$ is a retract of $X$ then the map $H_{n}(A)\to H_{n}(X)$ induced by the inclusion $A\subset X$ is injective.
\medskip

\textbf{Solution}. By the long exact sequence of a pair $(X, A)$, we have

$$\cdots \to H_{n}(A)\xrightarrow{i_{\ast}} H_{n}(X)\xrightarrow{j_{\ast}}H_{n}(X, A)\xrightarrow{\partial} H_{n-1}(A)\xrightarrow{i_{\ast}} H_{n-1}(X)\to \cdots \to H_{0}(X, A)\to 0.$$

Since $A$ is a retract of $X$, $H_{i}(X, A) = 0$ for all $i$ and the above reduces to

$$\cdots \to H_{n}(A)\xrightarrow{i_{\ast}} H_{n}(X)\xrightarrow{j_{\ast}}0\xrightarrow{\partial} H_{n-1}(A)\xrightarrow{i_{\ast}} H_{n-1}(X)\to \cdots \to H_{0}(X)\to 0.$$

Since $0\to A\xrightarrow{\alpha} B$ is exact iff $\ker \alpha = 0$, and the above sequence is exact, we conclude that $\ker i_{\ast} = 0$ and $i_{\ast}$ is injective. $\square$
\medskip

\textbf{Alternate}. If $A$ is a retract of $X$, then there exists $r: X\to A$ such that $r\circ i = \identity_{A}$. Since the induced identity is the identity, we have $(r\circ i)_{\ast} = r_{\ast}\circ i_{\ast} = \identity_{\ast} = \identity$, which implies $i_{\ast}$ is injective. $\blacksquare$
\bigskip
\bigskip

\textbf{15}. For an exact sequence $A\to B\to C\to D\to E$ show that $C = 0$ iff the map $A\to B$ is surjective and $D\to E$ is injective. Hence for a pair of spaces $(X, A)$, the inclusion $A\hookrightarrow X$ induces isomorphisms on all homology groups iff $H_{n}(X, A) = 0$ for all $n$.
\medskip

\textbf{Solution}. We first label the above sequence:

$$A\xrightarrow{\alpha} B\xrightarrow{\beta} C\xrightarrow{\gamma} D\xrightarrow{\delta} E.$$

If $C = 0$, then $\ker \beta = B$ and since the sequence is exact, $\im \alpha = \ker \beta = B$. So $\alpha: A\to B$ is surjective. We also have $0 = \im \gamma = \ker \delta$ so that $\delta:D\to E$ is injective. Conversely, if $\alpha: A\to B$ is surjective, then $B = \im \alpha = \ker \beta$ so that $\beta = 0$ and the above sequence reduces to

$$A\xrightarrow{\alpha} B\xrightarrow{\beta} 0\to C\xrightarrow{\gamma} D\xrightarrow{\delta} E.$$

Similarly, if $\delta: D\to E$ is injective, then $\im \gamma = \ker \delta = 0$ and the above sequence reduces to

$$A\xrightarrow{\alpha} B\xrightarrow{\beta} 0\to C\to 0\xrightarrow{\gamma} D\xrightarrow{\delta} E.$$

Hence $C = 0$. The implication being for the long exact sequence

$$\cdots \to H_{n}(A)\xrightarrow{i_{\ast}} H_{n}(X)\xrightarrow{j_{\ast}}H_{n}(X, A)\xrightarrow{\partial} H_{n-1}(A)\xrightarrow{i_{\ast}} H_{n-1}(X)\to \cdots \to H_{0}(X, A)\to 0$$

with $H_{n}(X, A) = 0$. At each dimension, the above sequence reduces to

$$\cdots\to  0\to H_{n}(A)\xrightarrow{i_{\ast}} H_{n}(X)\to 0\to \cdots $$

implying that the inclusion $i: A\hookrightarrow X$ induces isomorphisms $H_{n}(A)\cong H_{n}(X)$. $\blacksquare$
\bigskip
\bigskip

\textbf{16 (a)}. Show that $H_{0}(X, A) = 0$ iff $A$ meets each path-connected component of $X$.
\medskip

\textbf{Solution}. We have the long exact sequence

$$\cdots \to H_{1}(X, A)\to H_{0}(A)\xrightarrow{f_{\ast}} H_{0}(X) = \oplus_{i\in I} \Z\to H_{0}(X, A)\to 0$$

where $I$ the set of path-components of $X$. If $H_{0}(X, A) = 0$, then $f_{\ast}$ is surjective and the number of path components in $A$ is the same as in $X$. Since $A\subset X$, $A$ meets each path-connected component of $X$. 
\medskip

Conversely, if $A$ meets each path component of $X$, then $f_{\ast} = \identity_{\ast}$ is surjective and $H_{0}(X, A) = 0$. $\square$
\medskip

\textbf{(b)} Show that $H_{1}(X, A) = 0$ iff $H_{1}(A)\to H_{1}(X)$ is surjective and each path-component of $X$ contains at most one path-component of $A$.
\medskip

\textbf{Solution}. We have the long exact sequence

$$\cdots \to H_{1}(A)\xrightarrow{g_{\ast}} H_{1}(X)\to H_{1}(X, A)\to H_{0}(A)\xrightarrow{f_{\ast}} H_{0}(X)\to H_{0}(X, A)\to 0.$$

If $H_{1}(X, A) = 0$ then $g_{\ast}$ is surjective and $f_{\ast}$ is injective. The latter implies that no two path-components of $A$ can go to the same path component of $X$. That is, each path component of $X$ contains at most one path-component of $A$.
\medskip

Conversely, if $g_{\ast}$ is surjective and each path-component of $X$ contains at most one path component of $A$, the latter meaning $f_{\ast}$ is injective, by the previous exercise (2.1.15) $H_{1}(X, A) = 0$. $\blacksquare$
\bigskip
\bigskip

\textbf{19}. Compute the homology groups of the subspace of $I\times I$ consisting of the four boundary edges plus all points in the interior whose first coordinate is rational.
\medskip

\textbf{Solution}. Let $X$ denote this space. Since $X$ is path-connected, $H_{0}(X) = \Z$. Since there are no 2-cells, $H_{n}(X) = 0$ for all $n\geq 2$. We can describe this space as $X = \{(x, y)\in I\times I: x\in \Q\}\cup \{(x, y)\in I\times I: y\in \{0, 1\}\}$. Take two subspaces $A = \{(x, y)\in X: y < 3/4\}$ and $B = \{(x, y)\in X: y > 1/4\}$. These spaces retract to $y = 0$ and $y = 1$, respectively, so that they are both contractible. Their intersection is

$$A\cap B = \{(x, y)\in X: y\in (1/4, 3/4)\}\cong I\cap \Q$$

so that $H_{0}(A\cap B) = \Z^{|\Q|}$. Using the Mayer-Vietoris sequence on reduced homology, we have

$$0\to \tilde{H}_{1}(X)\to \tilde{H}_{0}(A\cap B)\to \tilde{H}_{0}(A)\oplus \tilde{H}_{0}(B)\to \tilde{H}_{0}(X)\to 0.$$

Since $A$ and $B$ are contractible, this becomes

$$0\to \tilde{H}_{1}(X)\to \tilde{H}_{0}(A\cap B)\to  0$$

so that $H_{1}(X) = \tilde{H}_{1}(X) = \tilde{H}_{0}(A\cap B) = \Z^{|\Q|-1} = \Z^{|\Q|}$. $\blacksquare$
\bigskip
\bigskip

\textbf{25}. Find an explicit, noninductive formula for the barycentric subdivision operator $S:C_{n}(X)\to C_{n}(X)$.
\medskip

\textbf{Solution}. In general we have the inductive operator taking $\sigma\in C_{n}(X)\to C_{n}(X)$ by

$$B_{p}(\sigma) = b(\sigma)\left(B_{p-1}(\partial\sigma)\right)$$

where $b$ is the barycenter of $\sigma$. For $n = 1$, we have
\begin{align*}
B[v_{0}, v_{1}] &= b([v_{0}, v_{1}])(B\partial [v_{0}, v_{1}]) = b([v_{0}, v_{1}])(B([v_{1}]-[v_{0}]))\\ &= b([v_{0}, v_{1}])([v_{1}]-[v_{0}]) = \left[\frac{v_{0}+v_{1}}{2}, v_{1}\right] - \left[\frac{v_{0}+v_{1}}{2}, v_{0}\right].
\end{align*}

For $n = 2$, we have
$$B[v_{0}, v_{1}, v_{2}] = b([v_{0}, v_{1}, v_{2}])(B\partial [v_{0}, v_{1}, v_{2}]) = b([v_{0}, v_{1}, v_{2}])(B([v_{1}, v_{2}]-[v_{0}, v_{2}] + [v_{0}, v_{1}]))$$
$$= b([v_{0}, v_{1}, v_{2}])\left(\left[\frac{v_{1}+v_{2}}{2}, v_{2}\right] - \left[\frac{v_{1}+v_{2}}{2}, v_{1}\right] - \left[\frac{v_{0}+v_{2}}{2}, v_{2}\right] + \left[\frac{v_{0}+v_{2}}{2}, v_{0}\right] + \left[\frac{v_{0}+v_{1}}{2}, v_{1}\right] - \left[\frac{v_{0}+v_{1}}{2}, v_{0}\right]\right)$$
$$= \left[\frac{v_{0}+v_{1}+v_{2}}{3},\frac{v_{1}+v_{2}}{2}, v_{2}\right] - \cdots + \left[\frac{v_{0}+v_{1}+v_{2}}{3}, \frac{v_{0}+v_{1}}{2}, v_{1}\right] - \left[\frac{v_{0}+v_{1}+v_{2}}{3}, \frac{v_{0}+v_{1}}{2}, v_{0}\right].$$

And now we can see a clear pattern where at each iteration, we add the barycenter of the $n$-th simplex to the image of the operator acting on the $(n-1)$-th simplex. We construct the non-inductive barycenter operator as 

$$B(\sigma_{n}) := \sum_{\pi \in S_{n+1}} \textnormal{sign}(\pi) \left[\frac{\sum_{i=0}^{n} v_{i}}{n+1}, \frac{\sum_{i=0}^{n-1} v^{\pi}_{i}}{n}, \ldots, \frac{\sum_{0}^{1} v^{\pi}_{i}}{1}, v^{\pi}_{0}\right]$$

where $S_{n}$ is the permutation group of $n$ vertices, $\textnormal{sign}(\pi)$ is the orientation of each permutation $\pi$, and where it applies, $v^{\pi}$ means the vertices that belong to the $(n-1)$-simplex of the $\pi$-th permutation. Note that in each element, we are summing over the $i$-th vertex of a permutation, and not the $i$-th index of $\sigma_{n}$. For example, in the last element, $v^{\pi}_{0}$ means the 0-th element of the $\pi$-th permutation, which could mean $v_{0}$, $v_{1}$, $v_{2}$, and so on. It does not strictly mean $v_{0}$. This is exemplified in our example for $n = 2$. $\blacksquare$
\bigskip
\bigskip

\textbf{29}. Show that $S^{1}\times S^{1}$ and $S^{1}\vee S^{1}\vee S^{2}$ have isomorphic homology groups in all dimensions, but their universal covering spaces do not.
\medskip

\textbf{Solution}. The homology groups are

\[H_{n}(S^{1}\times S^{1}) = \begin{cases} \Z^{2} & \textnormal{for } n=1\\ \Z & \textnormal{for } n = 0, 2\\
0 & \textnormal{for } n\geq3 \end{cases} = H_{n}(S^{1})\oplus H_{n}(S^{1})\oplus H_{n}(S^{2}) = H_{n}(S^{1}\vee S^{1}\vee S^{2}).\]

So $H_{n}(S^{1}\times S^{1}) = H_{n}(S^{1}\vee S^{1}\vee S^{2})$. The universal covering space of $S^{1}\times S^{1}$ is $\R^{2}$, which is contractible with homology groups $H_{n}(\R^{2}) = 0$ for all $n\neq 0$. The universal covering space of $S^{1}\vee S^{1}\vee S^{2}$ is the covering of $S^{1}\vee S^{1}$ with $S^{2}$ attached at each vertex.
\medskip

Incomplete...
\bigskip
\bigskip

\textbf{31}. Using the notation of the five-lemma, give an example where the maps $\alpha, \beta, \delta$, and $\epsilon$ are zero but $\gamma$ is nonzero. This can be done with the short exact sequences in which all the groups are either $\Z$ or 0.
\medskip

\textbf{Solution}. We use the same construction as in the reference:

\[\begin{tikzcd}
0 \arrow{r}\arrow{d}{\alpha = 0} & 0 \arrow{r}\arrow{d}{\beta = 0} & \Z \arrow{r}{\identity_{\ast}}\arrow{d}{\gamma = \identity_{\ast}} & \Z \arrow{r}\arrow{d}{\delta = 0} & 0 \arrow{d}{\epsilon = 0}\\
0 \arrow{r} & \Z \arrow{r}{\identity_{\ast}} & \Z \arrow{r} & 0 \arrow{r} & 0
\end{tikzcd}.\]
\medskip

Since $0\to \Z\to \Z\to 0$ is exact, both rows are exact. Commutativity of the squares can be checked easily. $\blacksquare$
\medskip

References: \href{https://cemulate.github.io/solutions_hatcher/e2-1-31.html}{1}.
\bigskip
\bigskip

\subsection{Computations and Applications}

\tab\textbf{1}. Prove the Brouwer fixed point theorem for maps $f:D^{n}\to D^{n}$ by applying degree theory to the map $S^{n}\to S^{n}$ that sends both the northern and southern hemispheres of $S^{n}$ to the southern hemisphere via $f$. [This was Brouwer's original proof.]
\medskip

\textbf{Solution}. Recall Brouwer's fixed point theorem states that for any continuous function $f$ mapping a compact convex set to itself, there is a point $x_{0}$ such that $f(x_{0}) = x_{0}$. Let $g: S^{n}\to S^{n}$ be the map described in the exercise. Since $g$ is not surjective, by degree property (b) (Page 134) $\deg g = 0$. If $g$ did not have any fixed points, then by degree property (g) $\deg g = (-1)^{n+1}$. So $g$ must have a fixed point, and it must be on the southern hemisphere of $S^{n}$. Any continuous map $f:D^{n}\to D^{n}$ can be described as the restriction to the southern hemisphere of some $g: S^{n}\to S^{n}$  That is, the fixed point of $g$ is the fixed point of $f$. $\blacksquare$
\medskip

References: \href{https://en.wikipedia.org/wiki/Brouwer_fixed-point_theorem}{1}, \href{https://pages.uoregon.edu/njp/635hw4solutions.pdf}{2}.
\bigskip
\bigskip

\textbf{3}. Let $f:S^{n}\to S^{n}$ be a map of degree zero. Show that there exist points $x, y\in S^{n}$ with $f(x) = x$ and $f(y) = -y$. Use this to show that if $F$ is a continuous vector field defined on the unit ball $D^{n}$ in $\R^{n}$ such that $F(x)\neq 0$ for all $x$, then there exists a point on $\partial D^{n}$ where $F$ points radially outward and another point on $\partial D^{n}$ where $F$ points radially inward.
\medskip

\textbf{Solution}. If $f$ did not have any fixed points, then by degree property (g) (Page 134) $\deg f = (-1)^{n+1}$. So $f$ must have a fixed point $x$ such that $f(x) = x$. Let $-\identity$ be the antipodal map. Then by degree property (d) $f\circ (-\identity) = -f$ has degree 0. So $-f$ must have a fixed point $y$ such that $-f(y) = y$.
\medskip

Since $F(x)\neq 0$, we have a well-defined normalised $\hat{F} = F/||F||\subseteq S^{n-1}$. Next define the degree 0 map (since the first map in the composition is not surjective)

$$G: S^{n-1}\hookrightarrow D^{n}\xrightarrow{\hat{F}} S^{n-1}.$$

By the first part, $\exists x, y$ such that $G(x) = x$ and $G(y) = -y$. Since the first map in the composition of $G$ is the inclusion map, we have $\hat{F}(x) = x$ and $\hat{F}(y) = -y$. In the original map, $F(x)$ is parallel and in the same direction as $x$, and $F(y)$ is parallel and in the opposite direction as $y$. I.e. $F(x)$ points radially outward and $F(y)$ points radially inward. $\blacksquare$
\medskip

References: \href{https://pages.uoregon.edu/njp/635hw4solutions.pdf}{1}.
\bigskip
\bigskip

\textbf{4}. Construct a surjective map $S^{n}\to S^{n}$ of degree zero, for each $n\geq 1$.
\medskip

\textbf{Solution}. Construct $f$ as the surjective map $h\circ g$ where $g: S^{n}\to D^{n}$ is the vertical projection and $h: D^{n}\to S^{n}$ is the quotient map, collapsing the border to a single point. Recall that the degree of $f$ is $d$ such that $f_{\ast}(\alpha) = d\alpha$ where $f_{\ast}: H_{n}(S^{n})\to H_{n}(S^{n})$. In our construction,

$$f_{\ast} = h_{\ast}\circ g_{\ast}: H_{n}(S^{n})\to H_{n}(D^{n}) = 0\to H_{n}(S^{n}).$$

Since $f_{\ast}$ passes through 0, it has degree zero. $\blacksquare$
\medskip

References: \href{https://pages.uoregon.edu/njp/635hw4solutions.pdf}{1}.
\bigskip
\bigskip

\textbf{6}. Show that every map $S^{n}\to S^{n}$ can be homotoped to have a fixed point if $n > 0$.
\medskip

\textbf{Solution}. If $f:S^{n}\to S^{n}$ has fixed points, we are done. Otherwise, if $f$ has no fixed points, $\deg f = (-1)^{n+1}$. If $n$ is even, $\deg f = -1$ and by degree property (c) (Page 134) we can homotope $f$ to a reflection, which has fixed points. If $n$ is odd, $\deg f = 1$ and we can homotope $f$ to the identity, which fixes every point. $\blacksquare$
\bigskip
\bigskip

\textbf{20}. For finite CW complexes $X$ and $Y$, show that $\chi(X\times Y) = \chi(X)\chi(Y)$.
\medskip

\textbf{Solution}. The Euler characteristic $\chi(X)$ is defined by $\sum_{n}(-1)^{n}c_{n}(X)$ where $c_{n}(X)$ is the number of $n$-cells in $X$. The $n$-cells in $X\times Y$ are the products of $i$-cells in $X$ and $j$-cells in $Y$ such that $i+j = n$. So

$$\chi(X\times Y) = \sum_{n} (-1)^{n}c_{n}(X\times Y) = \sum_{n} \sum_{i+j=n} (-1)^{i+j}c_{i}(X)\cdot c_{j}(Y) = \sum_{i}(-1)^{i}c_{i}(X)\cdot \sum_{j}(-1)^{j}c_{j}(Y) = \chi(X)\cdot \chi(Y).\quad \blacksquare$$
\bigskip

\textbf{21}. If a finite CW complex $X$ is the union of subcomplexes $A$ and $B$, show that $\chi(X) = \chi(A) + \chi(B) - \chi(A\cap B)$.
\medskip

\textbf{Solution}. This follows from $c_{n}(X) = c_{n}(A) + c_{n}(B) - c_{n}(A\cap B)$ where $A\cap B$ is a subcomplex consisting of the cells of $X$ both in $A$ and $B$. $\blacksquare$
\bigskip
\bigskip

\textbf{27}. The short exact sequences $0\to C_{n}(A)\to C_{n}(X)\to C_{n}(X, A)\to 0$ always split, but why does this not always yield splittings $H_{n}(X)\approx H_{n}(A)\oplus H_{n}(X, A)$?
\medskip

\textbf{Solution}. While the chain level $C(A)\to C(X)$ is injective, the induced homology $H(A)\to H(X)$ is not necessarily injective. For example, taking $A = S^{1}$ and $X = \R^{2}$ so that at the homology level, we have

$$H_{1}(A) = \Z \to H_{1}(X) = 0.$$

In a similar way, $H(X)\to H(X, A)$ is not necessarily surjective.  $\blacksquare$
\bigskip
\bigskip

\textbf{32}. For $SX$ the suspension of $X$, show by a Mayer-Vietoris sequence that there are isomorphisms $\tilde{H}_{n}(SX)\approx \tilde{H}_{n-1}(X)$ for all $n$.
\medskip

\textbf{Solution}. Recall that $SX = X\times I$. Taking $A = X\times [0, 3/4]$ and $B = X\times [1/4, 1]$, we have $SX = A\cup B$ and $X\simeq A\cap B$. We have the Mayer-Vietoris sequence

$$\cdots \to \tilde{H}_{n}(A)\oplus \tilde{H}_{n}(B)\to \tilde{H}_{n}(SX)\to \tilde{H}_{n-1}(X)\to \tilde{H}_{n-1}(A)\oplus \tilde{H}_{n-1}(B)\to \cdots$$

And since $A$ and $B$ are both contractible, $\tilde{H}_{n}(A) = \tilde{H}_{n}(B) = 0$. The above sequence then reduces to

$$\cdots \to 0\to \tilde{H}_{n}(SX)\to \tilde{H}_{n-1}(X)\to 0\to \cdots$$

implying $\tilde{H}_{n}(SX)\approx \tilde{H}_{n-1}(X)$ for all $n$. $\blacksquare$
\bigskip
\bigskip

\textbf{34}. [Deleted — see the errata for comments.]
\bigskip
\bigskip

\textbf{37}. Give an elementary derivation for the Mayer-Vietoris sequence in simplicial homology for a $\Delta$-complex $X$ decomposed as the union of subcomplexes $A$ and $B$.
\medskip

\textbf{Solution}. We want to show that

$$0\to C_{k}^{\Delta}(A\cap B)\xrightarrow{\alpha} C_{k}^{\Delta}(A)\oplus C_{k}^{\Delta}(B)\xrightarrow{\beta} C_{k}^{\Delta}(X)\to 0$$

is exact with $\alpha(x) = (x, -x)$ and $\beta(x, y) = x + y$. Let $\{x_{j}\}$ be the set of elements in the $\Delta$-complex of $X$. Then $\{x_{j}\} = \{a_{j}\}\cup \{b_{j}\}$
where $a_{j}, b_{j}$ are elements of respective subcomplexes $A$ and $B$. That is, we have

$$C_{k}^{\Delta}(X) = \Z \{x_{j}\},\quad C_{k}^{\Delta}(A) = \Z \{a_{j}\},\quad C_{k}^{\Delta}(B) = \Z \{b_{j}\}.$$

Finally, let $\{c_{j}\} = \{a_{j}\}\cap \{b_{j}\}$ so that $C_{k}^{\Delta}(A\cap B) = \Z \{c_{j}\}$. We want to show three things: $\ker \alpha = 0$, $\im \alpha = \ker \beta$, and $\im \beta = C_{k}^{\Delta}(X)$.
\medskip

In the first instance, if $x\in C_{k}^{\Delta}(A\cap B)$ and $\alpha(x) = (0, 0)$, then $x = 0$ so that $\ker \alpha = 0$, i.e. $\alpha$ is injective.
\medskip

Next, we have $\beta(\alpha(x)) = \beta(x, -x) = x + (-x) = 0$ so that $\im \alpha\subseteq \ker \beta$. Conversely, take $(x, y)\in \ker \beta\subseteq C_{k}^{\Delta}(A)\oplus C_{k}^{\Delta}(B)$. Then $0 = \beta(x, y) = x + y$ so that $x = -y$ and $(x, y) = (x, -x)\in \im \alpha$. So $\im \alpha\supseteq \ker \beta$ and $\im \alpha = \ker \beta$. Note here that $x = -y$ implies both $x$ and $y$ are in $C_{k}^{\Delta}(A)$ and $C_{k}^{\Delta}(B)$, i.e. they are both in $C_{k}^{\Delta}(A\cap B)$.
\medskip

In the last instance, let $x\in C_{k}^{\Delta}(X) = \Z\{x_{j}\} = \{a_{j}\}\cup \{b_{j}\}$. So

$$x = \sum_{j=1}^{n}n_{j}a_{j} + m_{j}b_{j}$$

for $n_{j},m_{j}\in \Z$. Then

$$\beta: C_{k}^{\Delta}(A)\oplus C_{k}^{\Delta}(B)\to C_{k}^{\Delta}(X)$$
$$(x, y)\mapsto x + y$$

applied to the individual components of $x$ gives

$$\beta\left(\sum_{j=1}^{n}n_{j}a_{j}, \sum_{j=1}^{n}m_{j}b_{j}\right) = \sum_{j=1}^{n}n_{j}a_{j} + m_{j}b_{j} = x$$

so that $x\in \im \beta$ and $\im \beta = C_{k}^{\Delta}(X)$. $\blacksquare$
\bigskip
\bigskip

\textbf{41}. For $X$ a finite CW complex and $F$ a field, show that the Euler characteristic $\chi(X)$ can also be computed by the formula $\chi(X) = \sum_{n}(-1)^{n} \dim H_{n}(X;F)$ the alternating sum of the dimensions of the vector spaces $H_{n}(X;F)$.
\medskip

\textbf{Solution}. There are two cases: when $\Char(F) = 0$ and when $\Char(F) = p$ where $p$ is prime. In the first case, the torsion of $F$ is empty and by the universal coefficient theorem for homology, we have

$$0\to H_{i}(X; \Z)\otimes F\to H_{i}(X; F)\to \Tor(H_{i-1}(X; \Z), F)\to 0.$$

Since $\Tor = 0$ when $F$ is torsion free, we have the isomorphism $H_{i}(X; \Z)\otimes F\cong H_{i}(X; F)$. Now since $X$ is a finite ($n$-dimensional) CW complex, for all $m > n$, $H_{m}(X) = 0$ and for all $i\leq n$ we have

$$H_{i}(X) = \Z^{\alpha_{i}}\oplus \sum_{k=1}^{m(i)} \Z_{\beta_{k}^{i}}.$$

Then

$$H_{i}(X; F)\cong H_{i}(X)\otimes F\cong F^{\alpha_{i}}$$

since $\Z_{n}\otimes F = 0$ and $\Z^{\alpha_{i}}$ is separated in the tensor product with $\Z\otimes F = F$. This isomorphism is given by

$$F^{\alpha_{i}}\to \Z^{\alpha_{i}}\otimes F\hookrightarrow \left(\Z^{\alpha_{i}}\oplus \sum_{k}^{m(i)} \Z_{\beta_{k}^{i}}\right)\otimes F\xrightarrow{\phi\otimes \identity} H_{i}(X)\otimes F\to H_{i}(X; F)$$
$$(v_{1},\ldots, v_{\alpha_{i}})\to \sum_{k}e_{k}\otimes v_{k}\to \sum_{k}e_{k}\otimes v_{k}\to \sum_{k}\phi(e_{k})\otimes v_{k}\to \sum_{k} v_{k}x_{k}$$

where $e_{k} = (0,\ldots, 1,\ldots, 0)$ on the $k$-th element, $x_{k}\in [\phi(e_{k})]$ is a fixed element, and $\phi: \Z^{\alpha_{i}}\oplus \sum_{k}^{m(i)}\Z_{\beta_{k}^{i}}\to H_{n}(X)$ is an isomorphism. So we have a vector space isomorphism and $\dim H_{n}(X; F) = \alpha_{n} = \textnormal{rank} (H_{n}(X))$. Then by definition of the Euler characteristic, we have

$$\chi(X) = \sum_{n}(-1)^{n}\textnormal{rank}(H_{n}(X)) = \sum_{n}(-1)^{n}\dim H_{n}(X; F).$$

Next we consider the case where $\Char(F) = p$ where $p$ is prime. Here we use the following lemma:

\[\Tor(\Z_{m}, F) = \begin{cases} F & \textnormal{if } p\ |\ m\\ 0 & \textnormal{otherwise}. \end{cases}\]

By this lemma, we have

$$\Tor(H_{i-1}(X), F) = \Tor\left(\Z^{\alpha_{i-1}}\oplus \sum_{k}^{m(i-1)}\Z_{\beta_{k}^{i-1}}, F\right) = \sum_{k}^{m(i-1)} \Tor(\Z_{\beta_{k}^{i-1}}, F) = \oplus_{p\ |\ \beta_{k}^{i-1}} F.$$

As in the first case, we have

$$0\to H_{i}(X; \Z)\otimes F\to H_{i}(X; F)\to \Tor(H_{i-1}(X; \Z), F)\to 0.$$

The maps are vector space homomorphisms. We expand the above to find

$$H_{i}(X; F)\cong H_{i}(X; \Z)\oplus \Tor(H_{i-1}(X; \Z), F).$$

Since

\[\Z_{m}\otimes F\cong F/mF = \begin{cases} F & \textnormal{if } p\ |\ m\\ 0 & \textnormal{otherwise}, \end{cases}\]

we have

$$H_{i}(X; F)\cong F^{\alpha_{i}}\oplus F^{\gamma_{i}}\oplus F^{\gamma_{i-1}}$$

where $\gamma_{i}$ is the number of times $p\ |\ m$ in the $i$-th homology group. Since $H_{k}(X) = 0$ for $k > n$, again using $\textnormal{rank}(H_{n}(X)) = \alpha_{n}$, we have

$$\sum_{n}(-1)^{n}\dim H_{n}(X; F) = \alpha_{0} + \gamma_{0} + \sum_{k=1}^{n} (-1)^{k}(\alpha_{k} + \gamma_{k} + \gamma_{k-1}) + (-1)^{n+1}(\gamma_{n}) = \sum_{k=0}^{n} (-1)^{n}\alpha_{k}$$

where the second last equality follows from the telescopic sum of $\gamma_{i}$. Since $\alpha_{k}$ is the rank of $H_{n}(X)$, the equality holds for the $\Char (F) = p$ where $p$ is prime. $\blacksquare$
\bigskip
\bigskip

\subsection{The Formal Viewpoint}

\subsection*{Additional Topics}
\addcontentsline{toc}{section}{\protect\numberline{}Additional Topics}

\subsubsection*{2.A. Homology and Fundamental Group}
\addcontentsline{toc}{subsection}{\protect\numberline{}2.A. Homology and Fundamental Group}

\tab No exercises in this subsection.
\bigskip
\bigskip

\subsubsection*{2.B. Classical Applications}
\addcontentsline{toc}{subsection}{\protect\numberline{}2.B. Classical Applications}

\tab \textbf{4}. In the unit sphere $S^{p+q-1}\subset \R^{p+1}$ let $S^{p-1}$ and $S^{q-1}$ be the subspheres consisting of points whose last $q$ and first $p$ coordinates are zero, respectively.
\medskip

\textbf{(a)} Show that $S^{p+q-1} - S^{p-1}$ deformation retracts onto $S^{q-1}$, and is in fact homeomorphic to $S^{q-1}\times \R^{p}$.
\medskip

\textbf{Solution}. We can take the homeomorphism

$$\phi: S^{q-1}\times \R^{p}\to S^{p+q-1}-S^{p-1}$$
$$(s_{p+1},\ldots, s_{p+q}, v_{1},\ldots, v_{p})\mapsto \frac{(v_{1},\ldots, v_{p}, s_{p+1},\ldots, s_{p+q})}{\sqrt{s_{p+1}^{2}+\cdots + s_{p+q}^{2} + v_{1}^{2} + \cdots + v_{p}^{2}}}$$

where in the domain, $\mathbf{v}\in \R^{p}$ is a vector. Note here that the image is correct because we have a $p + q$ vector such that the last $q$ coordinates are \textit{not} zero. If they were zero, then it would imply the first $p$ coordinates \textit{and} the last $q$ coordinates in $S^{q-1}$ are zero, which implies we just have the zero vector. But the zero vector is not in any $S^{i}$. Clearly we have a continuous map (with continuous inverse), since the denominator in the image is never zero. Since $S^{q-1}\times \R^{p}$ deformation retracts to $S^{q-1}\times \{0\} = S^{q-1}$, by the above homeomorphism we have a deformation retraction from $S^{p+q-1}-S^{p-1}$ to $S^{q-1}$. $\square$
\medskip

\textbf{(b)} Show that $S^{p-1}$ and $S^{q-1}$ are not the boundaries of any pair of disjointly embedded disks $D^{p}$ and $D^{q}$ in $D^{p+q}$. [The preceding exercise may be useful.]
\medskip

\textbf{Solution}.  Let $D^{p}\cap D^{q} = \emptyset$ for $D^{p}$ and $D^{q}$ in $D^{p+q}$. The assumption of the question states that $S^{p-1} = D^{p}\cap S^{p+q-1}$ and/or $S^{q-1} = D^{q}\cap S^{p+q-1}$. We consider the case of $S^{p-1}$. We have

\[\begin{tikzcd}[remember picture]
S^{p-1} \subseteq & \hspace{-1cm} D^{p}\\
S^{p+q-1}\textnormal{\textbackslash} S^{q-1} \subseteq & \hspace{-1cm} D^{p+q}\textnormal{\textbackslash} D^{q}
\end{tikzcd}
\begin{tikzpicture}[overlay, remember picture]
\path (\tikzcdmatrixname-1-1) to node[midway,sloped]{$\subseteq$}
(\tikzcdmatrixname-2-1); \hspace{-0.4cm}
\path (\tikzcdmatrixname-1-2) to  node[midway,sloped]{$\subseteq$} 
(\tikzcdmatrixname-2-2);\hspace{0.55cm}
\end{tikzpicture}\quad \Longrightarrow\quad \begin{tikzcd}
H_{\ast}(S^{p-1})\arrow{r}\arrow{d}{\cong} & H_{\ast}(D^{p})\arrow{d}\\
H_{\ast}(S^{p+q-1}\textnormal{\textbackslash} S^{q-1})\arrow{r}{\cong} & H_{\ast}(D^{p+q}\textnormal{\textbackslash} D^{q}).
\end{tikzcd}\]

The top down inclusion of spaces comes from the assumption $S^{p-1} = D^{p}\cap S^{p+q-1}$. The top-down isomorphism in homology is from part (a), while the left-right isomorphism is from the previous exercise (2.B.3). But $H_{p-1}(S^{p-1}) = \Z$ implies $H_{p-1}(D^{p+q}\textnormal{\textbackslash} D^{q}) = \Z$ which contradicts $H_{p-1}(D^{p}) = 0$ (contractible). The same is true for the case with $q$. So $S^{p-1}$ and $S^{q-1}$ are not the boundaries of any pair of disjointly embedded disks $D^{p}$ and $D^{q}$. $\blacksquare$
\bigskip
\bigskip

\subsubsection*{2.C. Simplicial Approximation}
\addcontentsline{toc}{subsection}{\protect\numberline{}2.C. Simplicial Approximation}

\newpage

\section{Cohomology}

\subsection{Cohomology Groups}

\tab \textbf{9}. Show that if $f:S^{n}\to S^{n}$ has degree $d$ then $f^{\ast}: H^{n}(S^{n};G)\to H^{n}(S^{n};G)$ is multiplication by $d$.
\medskip

\textbf{Solution}. For the multiplication by $d$ homomorphism $d: \Z\to d\Z$, the dualized homomorphism $d^{\ast}:\Z^{\ast}\to \Z^{\ast}$ is also multiplication by $d$ for any group $G$ where $\Z^{\ast} = \Hom(\Z, G)$. Then since $f_{\ast}:H_{n}(S^{n};G)\to H_{n}(S^{n};G)$ is multiplication by $d$, so is $f_{\star}:\Hom(H_{n}(S^{n}), G)\to \Hom(H_{n}(S^{n}), G)$. By the Universal Coefficient Theorem, we have

\[\begin{tikzcd}
0 \arrow{r} & \Ext(H_{n-1}(S^{n}), G) \arrow{r} \arrow{d} & H^{n}(S^{n};G) \arrow{r} \arrow{d}{f^{\ast}} & \Hom(H_{n}(S^{n}), G) \arrow{r} \arrow{d}{f_{\star}} & 0\\
0 \arrow{r} & \Ext(H_{n-1}(S^{n}), G) \arrow{r} & H^{n}(S^{n};G) \arrow{r} & \Hom(H_{n}(S^{n}), G) \arrow{r} & 0.
\end{tikzcd}\]
\medskip

Since $\Ext(H_{n-1}(S^{n}), G) = \Ext(0, G) = 0$, it follows that $H^{n}(S^{n};G)\cong \Hom(H_{n}(S^{n}), G)$ and $f^{\ast} = f_{\star}$. That is, $f^{\ast}:H^{n}(S^{n};G)\to H^{n}(S^{n};G)$ is multiplication by $d$. $\blacksquare$
\bigskip
\bigskip

\textbf{10}. For the lens space $L_{m}(\ell_{1}, \ldots, \ell_{n})$ defined in Example 2.43, compute the cohomology groups using the cellular cochain complex and taking coefficients in $\Z$, $\Q$, $\Z_{m}$, and $\Z_{p}$ for $p$ prime. Verify that the answers agree with those given by the universal coefficient theorem.
\medskip

\textbf{Solution}. From Example 2.43 we have

$$0\to \Z\xrightarrow{0} \Z\xrightarrow{m} \Z\to \cdots \to \Z \xrightarrow{m}\Z \xrightarrow{0}\Z\to 0.$$

To shorten syntax, we denote the lens space by $X$.
\bigskip

\textbf{Case 1} $\mathbf{G = \Z}$.

First we dualize with $\Hom(\Z, \Z)$ to obtain the sequence

$$0\to \Z\xrightarrow{0} \Z\xrightarrow{m} \Z\to \cdots \to \Z \xrightarrow{m}\Z \xrightarrow{0}\Z\to 0.$$

This follows from $\Hom(\Z, \Z) = \Z$ and the multiplication by 0 and $m$ carrying through in the dualized maps. Then $H^{i}(X) = \ker_{i+1}/\im_{i}$ so that when $i$ is even $H^{i}(X) = \Z/\{m\Z\} = \Z_{m}$. When $i$ is odd, $H^{i}(X) = \{0\}/\{0\} = 0$. In the special case where $i = 0$ we have $H^{0}(X) = \Z/\{0\} = \Z$. The same is true for the special case $i = 2n-1$.

\[H^{i}(X) = \begin{cases} \Z & \textnormal{for } i=0, 2n-1\\
\Z_{m} & \textnormal{for } i\textnormal{ even}\\
0 & \textnormal{otherwise}. \end{cases}\]
\medskip

Using the universal coefficient theorem, we have

$$0\to \Ext(H_{i-1}(C), \Z)\to H^{i}(C;\Z)\to \Hom(H_{i}(C),\Z)\to 0.$$

When $i$ is even we have

$$0\to \Ext(\Z_{m}, \Z)\to H^{i}(C;\Z)\to \Hom(0,\Z)\to 0.$$

so that $H^{i} = \Ext(\Z_{m},\Z) = \Z_{m}$. Similarly, when $i$ is odd we have $H^{i} = \Hom(\Z_{m},\Z) = 0$. In the special case where $i = 0$ we have $H^{0}(C, \Z) = \Hom(H_{0}(C), \Z) = \Hom(\Z, \Z) = \Z$. When $i = 2n-1$ we have $H^{2n-1}(C, \Z) = \Ext(H_{2n-2}(C), \Z)\oplus\Hom(H_{2n-1}(C), \Z) = \Ext(0, \Z)\oplus\Hom(\Z, \Z) = \Z$. This verifies case 1 with $G = \Z$.
\bigskip

\textbf{Case 2} $\mathbf{G = \Q}$.

First we dualize with $\Hom(\Z, \Q)$ to obtain the sequence

$$0\to \Q\xrightarrow{0} \Q\xrightarrow{m} \Q\to \cdots \to \Q \xrightarrow{m}\Q \xrightarrow{0}\Q\to 0.$$

This follows from $\Hom(\Z, \Q) = \Q$ and the multiplication by 0 and $m$ carrying through in the dualized maps. Then $H^{i}(X) = \ker_{i+1}/\im_{i}$ so that when $i$ is even $H^{i}(X) = \Q/\{m\Q\} = \Q/\Q = 0$. When $i$ is odd, $H^{i}(X) = \{0\}/\{0\} = 0$. In the special case where $i = 0$ we have $H^{0}(X) = \Q/\{0\} = \Q$. The same is true for the special case $i = 2n-1$.

\[H^{i}(X) = \begin{cases} \Q & \textnormal{for } i=0, 2n-1\\
0 & \textnormal{otherwise}. \end{cases}\]
\medskip

Using the universal coefficient theorem, we have

$$0\to \Ext(H_{i-1}(C), \Q)\to H^{i}(C;\Q)\to \Hom(H_{i}(C),\Q)\to 0.$$

When $i$ is even we have

$$0\to \Ext(\Z_{m}, \Q)\to H^{i}(C;\Q)\to \Hom(0,\Q)\to 0.$$

so that $H^{i} = \Ext(\Z_{m},\Q) = 0$. Similarly, when $i$ is odd we have $H^{i} = \Hom(\Z_{m},\Q) = 0$. In the special case where $i = 0$ we have $H^{0}(C, \Q) = \Hom(H_{0}(C), \Q) = \Hom(\Z, \Q) = \Q$. When $i = 2n-1$ we have $H^{2n-1}(C, \Q) = \Ext(H_{2n-2}(C), \Q)\oplus\Hom(H_{2n-1}(C), \Q) = \Ext(0, \Q)\oplus\Hom(\Z, \Q) = \Q$. This verifies case 2 with $G = \Q$.
\bigskip

\textbf{Case 3} $\mathbf{G = \Z_{m}}$.

First we dualize with $\Hom(\Z, \Z_{m})$ to obtain the sequence

$$0\to \Z_{m}\xrightarrow{0} \Z_{m}\xrightarrow{m} \Z_{m}\to \cdots \to \Z_{m} \xrightarrow{m}\Z_{m} \xrightarrow{0}\Z_{m}\to 0.$$

This follows from $\Hom(\Z, \Z_{m}) = \Z_{m}$ and the multiplication by 0 and $m$ carrying through in the dualized maps. Then $H^{i}(X) = \ker_{i+1}/\im_{i}$ so that when $i$ is even $H^{i}(X) = \Z_{m}/\{m\Z_{m}\} = \Z_{m}/\{0\} = \Z_{m}$ since $m\Z_{m}$ takes each element to $0\mod m$. For the same reason, when $i$ is odd, $H^{i}(X) = Z_{m}/\{0\} = \Z_{m}$. In the special case where $i = 0$ we have $H^{0}(X) = \Z_{m}/\{0\} = \Z_{m}$. The same is true for the special case $i = 2n-1$.

\[H^{i}(X) = \begin{cases} \Z_{m} & \textnormal{for } 0\leq i\leq 2n-1\\
0 & \textnormal{otherwise}. \end{cases}\]
\medskip

Using the universal coefficient theorem, we have

$$0\to \Ext(H_{i-1}(C), \Z_{m})\to H^{i}(C;\Z_{m})\to \Hom(H_{i}(C),\Z_{m})\to 0.$$

When $i$ is even we have

$$0\to \Ext(\Z_{m}, \Z_{m})\to H^{i}(C;\Z_{m})\to \Hom(0,\Z_{m})\to 0.$$

so that $H^{i} = \Ext(\Z_{m},\Z_{m}) = \Z_{m}$. Similarly, when $i$ is odd we have $H^{i} = \Hom(\Z_{m},\Z_{m}) = \Z_{m}$. In the special case where $i = 0$ we have $H^{0}(C, \Z_{m}) = \Hom(H_{0}(C), \Z_{m}) = \Hom(\Z, \Z_{m}) = \Z_{m}$. When $i = 2n-1$ we have $H^{2n-1}(C, \Z_{m}) = \Ext(H_{2n-2}(C), \Z_{m})\oplus\Hom(H_{2n-1}(C), \Z_{m}) = \Ext(0, \Z_{m})\oplus\Hom(\Z, \Z_{m}) = \Z_{m}$. This verifies case 3 with $G = \Z_{m}$.
\bigskip

\textbf{Case 4} $\mathbf{G = \Z_{p}}$ \textbf{for} $\mathbf{p}$ \textbf{prime}.

First we dualize with $\Hom(\Z, \Z_{p})$ to obtain the sequence

$$0\to \Z_{p}\xrightarrow{0} \Z_{p}\xrightarrow{m} \Z_{p}\to \cdots \to \Z_{p} \xrightarrow{m}\Z_{p} \xrightarrow{0}\Z_{p}\to 0.$$

This follows from $\Hom(\Z, \Z_{p}) = \Z_{p}$ and the multiplication by 0 and $m$ carrying through in the dualized maps. Now we separate into two cases:
\medskip

\textbf{Case 4 (a)} $\mathbf{p\nmid m}$.

If $p\nmid m$ then $\textnormal{gcd}(p, m) = 1$ and multiplication by $m$ is an isomorphism. Then $H^{i}(X) = \ker_{i+1}/\im_{i}$ so that when $i$ is even $H^{i}(X) = \Z_{p}/\{\Z_{p}\} = 0$. When $i$ is odd, $H^{i}(X) = \{0\}/\{0\} = 0$. In the special case where $i = 0$ we have $H^{0}(X) = \Z_{p}/\{0\} = \Z_{p}$. The same is true for the special case $i = 2n-1$.

\[H^{i}(X) = \begin{cases} \Z_{p} & \textnormal{for } i = 0, 2n-1\\
0 & \textnormal{otherwise}. \end{cases}\]
\medskip

Using the universal coefficient theorem, we have

$$0\to \Ext(H_{i-1}(C), \Z_{p})\to H^{i}(C;\Z_{p})\to \Hom(H_{i}(C),\Z_{p})\to 0.$$

When $i$ is even we have

$$0\to \Ext(\Z_{m}, \Z_{p})\to H^{i}(C;\Z_{p})\to \Hom(0,\Z_{p})\to 0.$$

so that $H^{i} = \Ext(\Z_{m},\Z_{p}) = 0$ since $\textnormal{gcd}(p, m) = 1$. Similarly, when $i$ is odd we have $H^{i} = \Hom(\Z_{m},\Z_{p}) = 0$ since $\textnormal{gcd}(p, m) = 1$. In the special case where $i = 0$ we have $H^{0}(C, \Z_{p}) = \Hom(H_{0}(C), \Z_{p}) = \Hom(\Z, \Z_{p}) = \Z_{p}$. When $i = 2n-1$ we have $H^{2n-1}(C, \Z_{p}) = \Ext(H_{2n-2}(C), \Z_{p})\oplus\Hom(H_{2n-1}(C), \Z_{p}) = \Ext(0, \Z_{p})\oplus\Hom(\Z, \Z_{p}) = \Z_{p}$. This verifies case 4 (a) with $G = \Z_{p}$ for $p$ prime and $p\nmid m$.
\medskip

\textbf{Case 4 (b)} $\mathbf{p\mid m}$.

If $p\mid m$ then multiplication by $m$ is the zero map and the cohomology group is the same as in case 3 with $G = \Z_{m}$.

\[H^{i}(X) = \begin{cases} \Z_{p} & \textnormal{for } 0\leq i\leq 2n-1\\
0 & \textnormal{otherwise}. \end{cases}\]
\medskip

Using the universal coefficient theorem, we have

$$0\to \Ext(H_{i-1}(C), \Z_{p})\to H^{i}(C;\Z_{p})\to \Hom(H_{i}(C),\Z_{p})\to 0.$$

When $i$ is even we have

$$0\to \Ext(\Z_{m}, \Z_{p})\to H^{i}(C;\Z_{p})\to \Hom(0,\Z_{p})\to 0.$$

so that $H^{i} = \Ext(\Z_{m},\Z_{p}) = \Z_{p}$ since $\textnormal{gcd}(p, m) = p$. Similarly, when $i$ is odd we have $H^{i} = \Hom(\Z_{m},\Z_{p}) = \Z_{p}$ since $\textnormal{gcd}(p, m) = p$. The two special cases are the same as in case (a). This verifies case 4 (b) with $G = \Z_{p}$ for $p$ prime and $p\mid m$. $\blacksquare$
\bigskip
\bigskip

\subsection{Cup Product}

\subsection{Poincaré Duality}

\tab\textbf{16}. Show that $(\alpha\frown \varphi)\frown \psi = \alpha \frown (\varphi\smile \psi)$ for all $\alpha\in C_{k}(X;R)$, $\varphi\in C^{l}(X; R)$, and $\psi\in C^{m}(X;R)$. Deduce that cap product makes $H_{\ast}(X;R)$ a right $H^{\ast}(X;R)$-module.
\medskip

\textbf{Solution}. On the right, we have

$$\alpha\frown(\varphi\smile \psi) = (\varphi\smile \psi)(\alpha|_{[v_{0},\ldots,v_{\ell+m}]})\cdot \alpha|_{[v_{\ell+m},\ldots, v_{k}]}$$
$$=\varphi\left((\alpha|_{[v_{0},\ldots,v_{\ell+m}]})|_{[v_{0},\ldots,v_{\ell}]}\right)\cdot \psi\left((\alpha|_{[v_{0},\ldots,v_{\ell+m}]})|_{[v_{\ell},\ldots,v_{m}]}\right)\cdot \alpha|_{[v_{\ell+m},\ldots, v_{k}]}$$
$$=\varphi(\alpha|_{[v_{0},\ldots,v_{\ell}]})\cdot \psi(\alpha|_{[v_{\ell},\ldots,v_{\ell+m}]})\cdot \alpha|_{[v_{\ell+m},\ldots, v_{k}]}$$

where the final result is an element in $C_{k}\frown (C^{\ell}\smile C^{m})\to C_{k}\frown C^{\ell+m}\to C_{k-(\ell+m)} = C_{k-\ell-m}$. On the left, we have

$$(\alpha\frown\varphi)\frown \psi = \varphi(\alpha|_{[v_{0},\ldots,v_{\ell}]})\cdot \alpha|_{[v_{\ell},\ldots,v_{k}]} \frown \psi$$
$$= \varphi(\alpha|_{[v_{0},\ldots, v_{\ell}]})\cdot \psi\left((\alpha|_{[v_{\ell},\ldots,v_{k}]})|_{[v_{0},\ldots, v_{m}]}\right)\cdot (\alpha|_{[v_{\ell},\ldots,v_{k}]})|_{[v_{m},\ldots,v_{k-\ell}]}$$
$$= \varphi(\alpha|_{[v_{0},\ldots, v_{\ell}]})\cdot \psi(\alpha|_{[v_{\ell},\ldots,v_{\ell+m}]})\cdot \alpha|_{[v_{\ell+m},\ldots,v_{k}]}$$

where the final result is an element in $(C_{k}\frown C^{\ell})\frown C^{m}\to C_{k-\ell}\frown C^{m}\to C_{k-\ell-m}$. So both sides are the same and equality holds. Note that $(\alpha|_{[v_{\ell},\ldots,v_{k}]})|_{[v_{0},\ldots, v_{m}]} = \alpha|_{[v_{\ell},\ldots,v_{\ell+m}]}$ since we are taking $m - 0 = m$ vertices from $\ell$ to $k$. And similarly, $(\alpha|_{[v_{\ell},\ldots,v_{k}]})|_{[v_{m},\ldots,v_{k-\ell}]} = \alpha|_{[v_{\ell+m},\ldots,v_{k}]}$ since we are taking $k - (\ell + m)$ vertices.
\medskip

Now we show that cap product makes $H_{\ast}(X; R)$ a right $H^{\ast}(X; R)$-module (see reference for conditions). First, for $m_{1}, m_{2}\in H_{\ast}(X;R)$ and $r\in H^{\ast}(X; R)$, we have

$$(m_{1}+m_{2})\frown r = r\frown m_{1} + r\frown m_{2}$$

because $\frown$ is an $R$-bilinear homomorphism. Next, for $m\in H_{\ast}(X; R)$ and $r_{1}, r_{2}\in H^{\ast}(X;R)$, we have

$$m\frown (r_{1} + r_{2}) = (r_{1} + r_{2})(m)\cdot m = r_{1}(m)\cdot m + r_{2}(m)\cdot m = m\frown r_{1} + m\frown r_{2}.$$

The final two conditions are the identity $m\frown \mathds{1} = \mathds{1}(m)\cdot m = m$ and associativity, which was proven in the first part of this exercise. So $\frown$ makes $H_{\ast}(X; R)$ a right $H^{\ast}(X; R)$-module. $\blacksquare$
\medskip

References: \href{https://math.stackexchange.com/questions/144518/what-exactly-is-an-r-module}{1}.
\bigskip
\bigskip

\subsection*{Additional Topics}
\addcontentsline{toc}{section}{\protect\numberline{}Additional Topics}

\subsubsection*{3.A. Universal Coefficients for Homology}
\addcontentsline{toc}{subsection}{\protect\numberline{}3.A. Universal Coefficients for Homology}

\subsubsection*{3.B. The General Künneth Formula}
\addcontentsline{toc}{subsection}{\protect\numberline{}3.B. The General Künneth Formula}

\subsubsection*{3.C. H–Spaces and Hopf Algebras}
\addcontentsline{toc}{subsection}{\protect\numberline{}3.C. H–Spaces and Hopf Algebras}

\subsubsection*{3.D. The Cohomology of SO(n)}
\addcontentsline{toc}{subsection}{\protect\numberline{}3.D. The Cohomology of SO(n)}

\subsubsection*{3.E. Bockstein Homomorphisms}
\addcontentsline{toc}{subsection}{\protect\numberline{}3.E. Bockstein Homomorphisms}

\subsubsection*{3.F. Limits and Ext}
\addcontentsline{toc}{subsection}{\protect\numberline{}3.F. Limits and Ext}

\subsubsection*{3.G. Transfer Homomorphisms}
\addcontentsline{toc}{subsection}{\protect\numberline{}3.G. Transfer Homomorphisms}

\tab No exercises in this subsection.
\bigskip
\bigskip

\subsubsection*{3.H. Local Coefficients}
\addcontentsline{toc}{subsection}{\protect\numberline{}3.H. Local Coefficients}

\newpage

\section{Homotopy Theory}

\subsection{Homotopy Groups}

\tab \textbf{1}. Suppose a sum $f+'g$ pf maps $f, g:(I^{n}, \partial I^{n})\to (X, x_{0})$ is defined using a coordinate of $I^{n}$ other than the first coordinate as in the usual sum $f+g$. Verify the formula $(f+g) +' (h+k) = (f+'h) + (g+'k)$, and deduce that $f+'k\simeq f+k$ so the two sums agree on $\pi_{n}(X, x_{0})$, and also that $g+'h\simeq h + g$ so the addition is abelian.
\medskip

\textbf{Solution}. Recall the usual definition for addition is 

\[(f+g)(\mathbf{s}) = \begin{cases} f(2s_{1}, s_{2}\ldots s_{n}) & \textnormal{for } s_{1}\in [0, \frac{1}{2}]\\ g(2s_{1} - 1, s_{2},\ldots s_{n}) & \textnormal{for } s_{1}\in [\frac{1}{2}, 1]. \end{cases} \]

Without loss of generality, assume addition uses the second coordinate:

\[(f+g)(\mathbf{s}) = \begin{cases} f(s_{1}, 2s_{2},\ldots s_{n}) & \textnormal{for } s_{2}\in [0, \frac{1}{2}]\\ g(s_{1}, 2s_{2} - 1,\ldots s_{n}) & \textnormal{for } s_{2}\in [\frac{1}{2}, 1]. \end{cases}\]

On the left, we have

\[(f+g)(\mathbf{s}) = \begin{cases} f(2s_{1}, s_{2}\ldots s_{n}) & \textnormal{for } s_{1}\in [0, \frac{1}{2}]\\ g(2s_{1} - 1, s_{2},\ldots s_{n}) & \textnormal{for } s_{1}\in [\frac{1}{2}, 1] \end{cases}\quad \textnormal{and}\quad  (h+k)(\mathbf{s}) = \begin{cases} h(2s_{1}, s_{2}\ldots s_{n}) & \textnormal{for } s_{1}\in [0, \frac{1}{2}]\\ k(2s_{1} - 1, s_{2},\ldots s_{n}) & \textnormal{for } s_{1}\in [\frac{1}{2}, 1] \end{cases}
\]

so that

\[\left((f+g)+'(h+k)\right)(\mathbf{s}) = \begin{cases} f(2s_{1}, 2s_{2},\ldots s_{n}) & \textnormal{for } s_{1}\in [0, \frac{1}{2}] \textnormal{, and } s_{2}\in [0, \frac{1}{2}]\\ 
g(2s_{1} - 1, 2s_{2},\ldots s_{n}) & \textnormal{for } s_{1}\in [\frac{1}{2}, 1] \textnormal{, and } s_{2}\in [0, \frac{1}{2}]\\
h(2s_{1}, 2s_{2}-1,\ldots s_{n}) & \textnormal{for } s_{1}\in [0, \frac{1}{2}] \textnormal{, and } s_{2}\in [\frac{1}{2}, 1]\\
k(2s_{1} - 1, 2s_{2}-1,\ldots s_{n}) & \textnormal{for } s_{1}\in [\frac{1}{2}, 1] \textnormal{, and } s_{2}\in [\frac{1}{2}, 1]. \end{cases} \]

On the right, we have

\[(f+'h)(\mathbf{s}) = \begin{cases} f(s_{1}, 2s_{2},\ldots s_{n}) & \textnormal{for } s_{2}\in [0, \frac{1}{2}]\\ h(s_{1}, 2s_{2} - 1,\ldots s_{n}) & \textnormal{for } s_{2}\in [\frac{1}{2}, 1] \end{cases}\quad \textnormal{and}\quad  (g+'k)(\mathbf{s}) = \begin{cases} g(s_{1}, 2s_{2},\ldots s_{n}) & \textnormal{for } s_{2}\in [0, \frac{1}{2}]\\ k(s_{1}, 2s_{2} - 1,\ldots s_{n}) & \textnormal{for } s_{2}\in [\frac{1}{2}, 1] \end{cases}
\]

so that

\[\left((f+'h)+(g+'k)\right)(\mathbf{s}) = \begin{cases} f(2s_{1}, 2s_{2},\ldots s_{n}) & \textnormal{for } s_{2}\in [0, \frac{1}{2}] \textnormal{, and } s_{1}\in [0, \frac{1}{2}]\\ 
h(2s_{1}, 2s_{2}-1,\ldots s_{n}) & \textnormal{for } s_{2}\in [\frac{1}{2}, 1] \textnormal{, and } s_{1}\in [0, \frac{1}{2}]\\
g(2s_{1}-1, 2s_{2},\ldots s_{n}) & \textnormal{for } s_{2}\in [0, \frac{1}{2}] \textnormal{, and } s_{1}\in [\frac{1}{2}, 1]\\
k(2s_{1} - 1, 2s_{2}-1,\ldots s_{n}) & \textnormal{for } s_{2}\in [\frac{1}{2}, 1] \textnormal{, and } s_{1}\in [\frac{1}{2}, 1]. \end{cases} \]

I.e. both sides agree and equality holds. If we take $g = h = 0$, then we get

$$(f+g)+'(h+k) = (f+'h)+(g+'k)\Longrightarrow (f+0)+'(0+k) = (f+'0)+(0+'k)\Longrightarrow f+'k = f+k$$

so that both additions agree on $\pi_{n}$. And taking $f = k = 0$ we have $g +' h = h + g$ so the addition is abelian. $\blacksquare$
\bigskip
\bigskip

\subsection{Elementary Methods of Calculation}

\subsection{Connections with Cohomology}

\subsection*{Additional Topics}
\addcontentsline{toc}{section}{\protect\numberline{}Additional Topics}

\subsubsection*{4.A. Basepoints and Homotopy}
\addcontentsline{toc}{subsection}{\protect\numberline{}4.A. Basepoints and Homotopy}

\subsubsection*{4.B. The Hopf Invariant}
\addcontentsline{toc}{subsection}{\protect\numberline{}4.B. The Hopf Invariant}

\subsubsection*{4.C. Minimal Cell Structures}
\addcontentsline{toc}{subsection}{\protect\numberline{}4.C. Minimal Cell Structures}

\tab No exercises in this subsection.
\bigskip
\bigskip

\subsubsection*{4.D. Cohomology of Fiber Bundles}
\addcontentsline{toc}{subsection}{\protect\numberline{}4.D. Cohomology of Fiber Bundles}

\subsubsection*{4.E. The Brown Representability Theorem}
\addcontentsline{toc}{subsection}{\protect\numberline{}4.E. The Brown Representability Theorem}

\tab No exercises in this subsection.
\bigskip
\bigskip

\subsubsection*{4.F. Spectra and Homology Theories}
\addcontentsline{toc}{subsection}{\protect\numberline{}4.F. Spectra and Homology Theories}

\subsubsection*{4.G. Gluing Constructions}
\addcontentsline{toc}{subsection}{\protect\numberline{}4.G. Gluing Constructions}

\subsubsection*{4.H. Eckmann-Hilton Duality}
\addcontentsline{toc}{subsection}{\protect\numberline{}4.H. Eckmann-Hilton Duality}

\subsubsection*{4.I. Stable Splittings of Spaces}
\addcontentsline{toc}{subsection}{\protect\numberline{}4.I. Stable Splittings of Spaces}

\subsubsection*{4.J. The Loopspace of a Suspension}
\addcontentsline{toc}{subsection}{\protect\numberline{}4.J. The Loopspace of a Suspension}

\subsubsection*{4.K. The Dold-Thom Theorem}
\addcontentsline{toc}{subsection}{\protect\numberline{}4.K. The Dold-Thom Theorem}

\subsubsection*{4.L. Steenrod Squares and Powers}
\addcontentsline{toc}{subsection}{\protect\numberline{}4.L. Steenrod Squares and Powers}

\end{document}