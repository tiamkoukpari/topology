\documentclass{article}
\usepackage[utf8]{inputenc}
\usepackage[english]{babel}
\usepackage[margin=0.8in]{geometry}
\usepackage[symbol]{footmisc}
\usepackage[titles]{tocloft}
\usepackage[shortlabels]{enumitem}

\usepackage{amsmath, amssymb, MnSymbol, wasysym, varwidth, enumitem, array, caption, floatrow, wasysym, tikz, tikz-cd, graphicx, hyperref, dsfont, tabto}

\renewcommand{\cftsecfont}{\normalfont\mdseries}
\usepackage{hyperref}
\hypersetup{
    colorlinks,
    citecolor=blue,
    filecolor=blue,
    linkcolor=blue,
    urlcolor=blue
}

\newcommand{\R}{\mathbb{R}}
\newcommand{\C}{\mathbb{C}}
\newcommand{\Z}{\mathbb{Z}}
\newcommand{\Q}{\mathbb{Q}}
\newcommand{\I}{[0, 1]}
\newcommand{\h}{\simeq}
\newcommand{\identity}{\mathds{1}}
\newcommand{\Hom}{\textnormal{Hom}}
\newcommand{\Ext}{\textnormal{Ext}}
\newcommand{\Tor}{\textnormal{Tor}}
\newcommand{\Char}{\textnormal{Char}}
\newcommand{\im}{\textnormal{im}}

\setcounter{section}{-1}

\title{Hatcher's Algebraic Topology - Solutions}
\author{Institute for Pure and Applied Mathematics (IMPA)}
\date{Written by: Tiam Koukpari}

\begin{document}
\maketitle

Trying to collect the fragmented sets of solutions into one file. Here is the sequence of requisites needed for this topic:
\bigskip

\[\begin{tikzcd}
& \textnormal{Differential Forms (de Rham Cohomology)} \arrow[d] & \\
& \textnormal{Analysis in } \R^{n} \subseteq \textnormal{Topology} \subseteq \textnormal{Manifolds} \arrow[rrd] \arrow[u] & \\
\textnormal{Set Theory} \arrow[ru] \arrow[rd] & & & \textnormal{Algebraic Topology} \\
& \textnormal{Groups} \subseteq \textnormal{Rings} \subseteq \textnormal{Fields} \arrow[r]\arrow[d] & \textnormal{Linear Algebra} \arrow[ru] & \\
& \textnormal{Homological Algebra \& Categories} \arrow[u]
\end{tikzcd}\]
\bigskip
\bigskip

If you find any mistakes, please email \href{mailto:tiam.koukpari@impa.br}{tiam.koukpari@impa.br}.
\newpage

\tableofcontents
\newpage

\section{Some Underlying Geometric Notions}

\tab\textbf{1}. Construct an explicit deformation retraction of the torus with one point deleted onto a graph consisting of two circles intersecting in a point, namely, longitude and meridian circles of the torus.
\medskip

\textbf{Solution}. It is useful to visualize the torus with
\usetikzlibrary{decorations.markings,positioning}

\[\begin{tikzpicture}[node distance=2cm]
\coordinate (top-right) {};
\coordinate[left=of top-right] (top-left) {};
\coordinate[below=of top-right] (bottom-left) {};
\coordinate[below=of top-left] (bottom-right) {};
\begin{scope}[decoration={markings, mark=at position 0.5 with {\arrow{<}}}]
\draw[postaction={decorate}] (top-right) -- node[auto,swap] {$v$} (top-left);
\draw[postaction={decorate}] (bottom-right) -- node[auto] {$w$} (top-left);
\end{scope}
\begin{scope}[decoration={markings, mark=at position 0.5 with {\arrow{>}}}]
\draw[postaction={decorate}] (bottom-right) -- node[auto,swap] {$v$} (bottom-left);
\draw[postaction={decorate}] (top-right) -- node[auto] {$w$} (bottom-left);
\end{scope}
\end{tikzpicture}\]

To form a torus from the above, fold the shape to connect $v$ with itself, creating two copies of $S^{1}$ on $w$. Then fold the shape to connect $w$ with itself, joining the two copies of $S^{1}$ on $w$ and creating another $S^{1}$ on $v$...

$\blacksquare$
\bigskip
\bigskip

\textbf{2}. Construct an explicit deformation retraction of $\R^{n} - \{0\}$ onto $S^{n-1}$.
\medskip

\textbf{Solution}. Construct

$$f_{t}(\mathbf{x}) = (1-t)\mathbf{x} + t\frac{\mathbf{x}}{|\mathbf{x}|}.$$

Then $f_{0}(\mathbf{x}) = \mathbf{x}$ so that $f_{0} = \identity$, $f_{1}(\mathbf{x}) = \mathbf{x}/|\mathbf{x}|$ so that $f_{1} = S^{n-1}$, and $f_{t}|S^{n-1} = \identity$. The function is a straight, continuous line from $\mathbf{x}$ to a normalized $\mathbf{x}$, i.e. on the $(n-1)$-sphere. The function is continuous since $\{0\}$ is not in its domain. $\blacksquare$
\bigskip
\bigskip

\textbf{3}. (a) Show that the composition of homotopy equivalence $X\to Y$ and $Y\to Z$ is a homotopy equivalence $X\to Z$. Deduce that homotopy equivalence is an equivalence relation.
\medskip

\textbf{Solution}. $\square$
\medskip

(b) Show that the relation of homotopy among maps $X\to Y$ is an equivalence relation.
\medskip

\textbf{Solution}. $\square$
\medskip

(c) Show that a map homotopic to a homotopy equivalence is a homotopy equivalence.
\medskip

\textbf{Solution}. $\blacksquare$
\bigskip
\bigskip

\textbf{4}.
\newpage

\section{The Fundamental Group}

\subsection{Basic Constructions}

\subsection{Van Kampen's Theorem}

\subsection{Covering Spaces}

\subsection*{Additional Topics}
\addcontentsline{toc}{section}{\protect\numberline{}Additional Topics}

\subsubsection*{1.A. Graphs and Free Groups}
\addcontentsline{toc}{subsection}{\protect\numberline{}1.A. Graphs and Free Groups}

\subsubsection*{1.B. K(G,1) Spaces and Graphs of Groups}
\addcontentsline{toc}{subsection}{\protect\numberline{}1.B. K(G,1) Spaces and Graphs of Groups}

\newpage

\section{Homology}

\subsection{Simplicial and Singular Homology}

\subsection{Computations and Applications}

\subsection{The Formal Viewpoint}

\subsection*{Additional Topics}
\addcontentsline{toc}{section}{\protect\numberline{}Additional Topics}

\subsubsection*{2.A. Homology and Fundamental Group}
\addcontentsline{toc}{subsection}{\protect\numberline{}2.A. Homology and Fundamental Group}

\subsubsection*{2.B. Classical Applications}
\addcontentsline{toc}{subsection}{\protect\numberline{}2.B. Classical Applications}

\subsubsection*{2.C. Simplicial Approximation}
\addcontentsline{toc}{subsection}{\protect\numberline{}2.C. Simplicial Approximation}

\newpage

\section{Cohomology}

\subsection{Cohomology Groups}

\subsection{Cup Product}

\subsection{Poincaré Duality}

\subsection*{Additional Topics}
\addcontentsline{toc}{section}{\protect\numberline{}Additional Topics}

\subsubsection*{3.A. Universal Coefficients for Homology}
\addcontentsline{toc}{subsection}{\protect\numberline{}3.A. Universal Coefficients for Homology}

\subsubsection*{3.B. The General Künneth Formula}
\addcontentsline{toc}{subsection}{\protect\numberline{}3.B. The General Künneth Formula}

\subsubsection*{3.C. H–Spaces and Hopf Algebras}
\addcontentsline{toc}{subsection}{\protect\numberline{}3.C. H–Spaces and Hopf Algebras}

\subsubsection*{3.D. The Cohomology of SO(n)}
\addcontentsline{toc}{subsection}{\protect\numberline{}3.D. The Cohomology of SO(n)}

\subsubsection*{3.E. Bockstein Homomorphisms}
\addcontentsline{toc}{subsection}{\protect\numberline{}3.E. Bockstein Homomorphisms}

\subsubsection*{3.F. Limits and Ext}
\addcontentsline{toc}{subsection}{\protect\numberline{}3.F. Limits and Ext}

\subsubsection*{3.G. Transfer Homomorphisms}
\addcontentsline{toc}{subsection}{\protect\numberline{}3.G. Transfer Homomorphisms}

\subsubsection*{3.H. Local Coefficients}
\addcontentsline{toc}{subsection}{\protect\numberline{}3.H. Local Coefficients}

\newpage

\section{Homotopy Theory}

\subsection{Homotopy Groups}

\subsection{Elementary Methods of Calculation}

\subsection{Connections with Cohomology}

\subsection*{Additional Topics}
\addcontentsline{toc}{section}{\protect\numberline{}Additional Topics}

\subsubsection*{4.A. Basepoints and Homotopy}
\addcontentsline{toc}{subsection}{\protect\numberline{}4.A. Basepoints and Homotopy}

\subsubsection*{4.B. The Hopf Invariant}
\addcontentsline{toc}{subsection}{\protect\numberline{}4.B. The Hopf Invariant}

\subsubsection*{4.C. Minimal Cell Structures}
\addcontentsline{toc}{subsection}{\protect\numberline{}4.C. Minimal Cell Structures}

\subsubsection*{4.D. Cohomology of Fiber Bundles}
\addcontentsline{toc}{subsection}{\protect\numberline{}4.D. Cohomology of Fiber Bundles}

\subsubsection*{4.E. The Brown Representability Theorem}
\addcontentsline{toc}{subsection}{\protect\numberline{}4.E. The Brown Representability Theorem}

\subsubsection*{4.F. Spectra and Homology Theories}
\addcontentsline{toc}{subsection}{\protect\numberline{}4.F. Spectra and Homology Theories}

\subsubsection*{4.G. Gluing Constructions}
\addcontentsline{toc}{subsection}{\protect\numberline{}4.G. Gluing Constructions}

\subsubsection*{4.H. Eckmann-Hilton Duality}
\addcontentsline{toc}{subsection}{\protect\numberline{}4.H. Eckmann-Hilton Duality}

\subsubsection*{4.I. Stable Splittings of Spaces}
\addcontentsline{toc}{subsection}{\protect\numberline{}4.I. Stable Splittings of Spaces}

\subsubsection*{4.J. The Loopspace of a Suspension}
\addcontentsline{toc}{subsection}{\protect\numberline{}4.J. The Loopspace of a Suspension}

\subsubsection*{4.K. The Dold-Thom Theorem}
\addcontentsline{toc}{subsection}{\protect\numberline{}4.K. The Dold-Thom Theorem}

\subsubsection*{4.L. Steenrod Squares and Powers}
\addcontentsline{toc}{subsection}{\protect\numberline{}4.L. Steenrod Squares and Powers}

\end{document}