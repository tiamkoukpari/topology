\documentclass{article}
\usepackage[utf8]{inputenc}
\usepackage[english]{babel}
\usepackage[margin=0.8in]{geometry}
\usepackage[symbol]{footmisc}
\usepackage[titles]{tocloft}
\usepackage[shortlabels]{enumitem}

\usepackage{amsmath, amssymb, MnSymbol, wasysym, varwidth, enumitem, array, caption, floatrow, wasysym, tikz, tikz-cd, graphicx, hyperref, dsfont, tabto}

\renewcommand{\cftsecfont}{\normalfont\mdseries}
\usepackage{hyperref}
\hypersetup{
    colorlinks,
    citecolor=blue,
    filecolor=blue,
    linkcolor=blue,
    urlcolor=blue
}

\newcommand{\R}{\mathbb{R}}
\newcommand{\C}{\mathbb{C}}
\newcommand{\Z}{\mathbb{Z}}
\newcommand{\Q}{\mathbb{Q}}
\newcommand{\I}{[0, 1]}
\newcommand{\h}{\simeq}
\newcommand{\identity}{\mathds{1}}
\newcommand{\Hom}{\textnormal{Hom}}
\newcommand{\Ext}{\textnormal{Ext}}
\newcommand{\Tor}{\textnormal{Tor}}
\newcommand{\Char}{\textnormal{Char}}
\newcommand{\im}{\textnormal{im}}

\setcounter{section}{-1}

\title{Hatcher's Algebraic Topology - Solutions}
\author{Institute for Pure and Applied Mathematics (IMPA)}
\date{Written by: Tiam Koukpari}

\begin{document}
\maketitle

Trying to collect the fragmented sets of solutions into one file. Here is the sequence of requisites needed for this topic:
\bigskip

\[\begin{tikzcd}
& \textnormal{Differential Forms (de Rham Cohomology)} \arrow[d] & \\
& \textnormal{Analysis in } \R^{n} \subseteq \textnormal{Topology} \subseteq \textnormal{Manifolds} \arrow[rrd] \arrow[u] & \\
\textnormal{Set Theory} \arrow[ru] \arrow[rd] & & & \textnormal{Algebraic Topology} \\
& \textnormal{Groups} \subseteq \textnormal{Rings} \subseteq \textnormal{Fields} \arrow[r]\arrow[d] & \textnormal{Linear Algebra} \arrow[ru] & \\
& \textnormal{Homological Algebra \& Categories} \arrow[u]
\end{tikzcd}\]
\bigskip
\bigskip

References, if used, are included at the end of each exercise.
\medskip

If you find any mistakes or if you want to submit a solution, please email \href{mailto:tiam.koukpari@impa.br}{tiam.koukpari@impa.br}. The remaining problems are:
\medskip

\textbf{Chapter 0}:
\medskip

1, 3, 4, 5, 6, 7, 8, 9, 10, 11, 12, 13, 14, 15, 16, 17, 18, 19, 20, 21, 22, 23, 24, 25, 26, 27, 28, 29
\medskip

\textbf{Chapter 1}:
\medskip

1.1, 1.2, 1.3, 1.4, 1.5, 1.6, 1.7, 1.8, 1.9, 1.10, 1.11, 1.12, 1.13, 1.14, 1.15, 1.16, 1.17, 1.18, 1.19, 1.20

2.1, 2.2, 2.3, 2.4, 2.5, 2.6, 2.7, 2.8, 2.9, 2.10, 2.11, 2.12, 2.13, 2.14, 2.15, 2.16, 2.17, 2.18, 2.19, 2.20, 2.21, 2.22

3.1, 3.2, 3.3, 3.4, 3.5, 3.6, 3.7, 3.8, 3.9, 3.10, 3.11, 3.12, 3.13, 3.14, 3.15, 3.16, 3.17, 3.18, 3.19, 3.20, 3.21, 3.22, 3.23, 3.24, 3.25, 3.26, 3.27, 3.28, 3.29, 3.30, 3.31, 3.32, 3.33

A.1, A.2, A.3, A.4, A.5, A.6, A.7, A.8, A.9, A.10, A.11, A.12, A.13, A.14

B.1, B.2, B.3, B.4, B.5, B.6, B.7, B.8, B.9
\medskip

\textbf{Chapter 2}:
\medskip

1.2, 1.3, 1.4, 1.5, 1.6, 1.7, 1.8, 1.9, 1.10, 1.11, 1.12, 1.13, 1.14, 1.15, 1.16, 1.17, 1.18, 1.19, 1.20, 1.21, 1.22, 1.23, 1.24, 1.26, 1.27, 1.28, 1.29, 1.30, 1.31

2.1, 2.2, 2.3, 2.4, 2.5, 2.6, 2.7, 2.8, 2.9, 2.10, 2.11, 2.12, 2.13, 2.14, 2.15, 2.16, 2.17, 2.18, 2.19, 2.20, 2.21, 2.22 2.23, 2.24, 2.25, 2.26, 2.27, 2.28, 2.29, 2.30, 2.31, 2.32, 2.33, 2.34, 2.35, 2.36, 2.37, 2.38, 2.39, 2.40, 2.41, 2.42, 2.43

3.1, 3.2, 3.3, 3.4

B.1, B.2, B.3, B.4, B.5, B.6, B.7, B.8, B.9, B.10, B.11

C.1, C.2, C.3, C.4, C.5, C.6, C.7, C.8, C.9
\medskip

\textbf{Chapter 3}:
\medskip

1.1, 1.2, 1.3, 1.4, 1.5, 1.6, 1.7, 1.8, 1.9, 1.10, 1.11, 1.12, 1.13

2.1, 2.2, 2.3, 2.4, 2.5, 2.6, 2.7, 2.8, 2.9, 2.10, 2.11, 2.12, 2.13, 2.14, 2.15, 2.16, 2.17, 2.18

3.1, 3.2, 3.3, 3.4, 3.5, 3.6, 3.7, 3.8, 3.9, 3.10, 3.11, 3.12, 3.13, 3.14, 3.15, 3.16, 3.17, 3.18, 3.19, 3.20, 3.21, 3.22, 3.23, 3.24, 3.25, 3.26, 3.27, 3.28, 3.29, 3.30, 3.31, 3.32, 3.33, 3.34, 3.35

A.1, A.2, A.3, A.4, A.5, A.6

B.1, B.2, B.3, B.4, B.5

C.1, C.2, C.3, C.4, C.5, C.6, C.7, C.8, C.9, C.10, C.11, C.12, C.13, C.14, C.15, C.16

D.1, D.2, D.3

E.1, E.2, E.3, E.4

F.1, F.2, F.3, F.4, F.5, F.6, F.7, F.8, F.9

H.1, H.2, H.3, H.4, H.5, H.6
\medskip

\textbf{Chapter 4}:
\medskip

1.1, 1.2, 1.3, 1.4, 1.5, 1.6, 1.7, 1.8, 1.9, 1.10, 1.11, 1.12, 1.13, 1.14, 1.15, 1.16, 1.17, 1.18, 1.19, 1.20, 1.21, 1.22, 1.23

2.1, 2.2, 2.3, 2.4, 2.5, 2.6, 2.7, 2.8, 2.9, 2.10, 2.11, 2.12, 2.13, 2.14, 2.15, 2.16, 2.17, 2.18, 2.19, 2.20, 2.21, 2.22 2.23, 2.24, 2.25, 2.26, 2.27, 2.28, 2.29, 2.30, 2.31, 2.32, 2.33, 2.34, 2.35, 2.36, 2.37, 2.38, 2.39

3.1, 3.2, 3.3, 3.4, 3.5, 3.6, 3.7, 3.8, 3.9, 3.10, 3.11, 3.12, 3.13, 3.14, 3.15, 3.16, 3.17, 3.18, 3.19, 3.20, 3.21, 3.22, 3.23, 3.24

A.1, A.2, A.3, A.4, A.5

B.1, B.2

D.1, D.2, D.3, D.4, D.5, D.6, D.7, D.8, D.9, D.10

F.1, F.2, F.3

G.1, G.2, G.3, G.4

H.1, H.2, H.3, H.4

I.1 I.2, I.3

J.1

K.1, K.2, K.3, K.4, K.5, K.6

L.1, L.2, L.3, L.4, L.5
\newpage

\tableofcontents
\newpage

\section{Some Underlying Geometric Notions}

\tab\textbf{1}. Construct an explicit deformation retraction of the torus with one point deleted onto a graph consisting of two circles intersecting in a point, namely, longitude and meridian circles of the torus.
\medskip

\textbf{Solution}. It is useful to visualize the torus with
\usetikzlibrary{decorations.markings,positioning}

\[\begin{tikzpicture}[node distance=2cm]
\coordinate (top-right) {};
\coordinate[left=of top-right] (top-left) {};
\coordinate[below=of top-right] (bottom-left) {};
\coordinate[below=of top-left] (bottom-right) {};
\begin{scope}[decoration={markings, mark=at position 0.5 with {\arrow{<}}}]
\draw[postaction={decorate}] (top-right) -- node[auto,swap] {$v$} (top-left);
\draw[postaction={decorate}] (bottom-right) -- node[auto] {$w$} (top-left);
\end{scope}
\begin{scope}[decoration={markings, mark=at position 0.5 with {\arrow{>}}}]
\draw[postaction={decorate}] (bottom-right) -- node[auto,swap] {$v$} (bottom-left);
\draw[postaction={decorate}] (top-right) -- node[auto] {$w$} (bottom-left);
\end{scope}
\end{tikzpicture}\]

To form a torus from the above, fold the shape to connect $v$ with itself, creating two copies of $S^{1}$ on $w$. Then fold the shape to connect $w$ with itself, joining the two copies of $S^{1}$ on $w$ and creating another $S^{1}$ on $v$...

$\blacksquare$
\bigskip
\bigskip

\textbf{2}. Construct an explicit deformation retraction of $\R^{n} - \{0\}$ onto $S^{n-1}$.
\medskip

\textbf{Solution}. Construct

$$f_{t}(\mathbf{x}) = (1-t)\mathbf{x} + t\frac{\mathbf{x}}{|\mathbf{x}|}.$$

Then $f_{0}(\mathbf{x}) = \mathbf{x}$ so that $f_{0} = \identity$, $f_{1}(\mathbf{x}) = \mathbf{x}/|\mathbf{x}|$ so that $f_{1} = S^{n-1}$, and $f_{t}|S^{n-1} = \identity$. The function is a straight, continuous line from $\mathbf{x}$ to a normalized $\mathbf{x}$, i.e. on the $(n-1)$-sphere. The function is continuous since $\{0\}$ is not in its domain. $\blacksquare$
\bigskip
\bigskip

\textbf{3}. (a) Show that the composition of homotopy equivalence $X\to Y$ and $Y\to Z$ is a homotopy equivalence $X\to Z$. Deduce that homotopy equivalence is an equivalence relation.
\medskip

\textbf{Solution}. Let $f:X\to Y$ be a homotopy equivalence and $f^{-1}:Y\to X$ its inverse. Similarly, let $g:Y\to Z$ be a homotopy equivalence and $g^{-1}:Z\to Y$ its inverse. Construct $h := g\circ f$ and $h^{-1} := f^{-1}\circ g^{-1}$. We want to show that $h\circ h^{-1}\simeq \identity$:

$$h\circ h^{-1} = g\circ f\circ f^{-1}\circ g^{-1}\simeq g\circ \identity\circ g^{-1} = g\circ g^{-1}\simeq \identity.$$

(b) Show that the relation of homotopy among maps $X\to Y$ is an equivalence relation.
\medskip

\textbf{Solution}. $\square$
\medskip

(c) Show that a map homotopic to a homotopy equivalence is a homotopy equivalence.
\medskip

\textbf{Solution}. $\blacksquare$
\bigskip
\bigskip

\textbf{4}. A \textbf{deformation retraction in the weak sense} of a space $X$ to a subspace $A$ is a homotopy $f_{t}: X\to X$ such that $f_{0} = \identity$, $f_{1}(X)\subset A$, and $f_{t}(A)\subset A$ for all $t$. Show that if $X$ deformation retracts to $A$ in this weak sense, then the inclusion $A\hookrightarrow X$ is a homotopy equivalence.
\medskip

\textbf{Solution}. $\blacksquare$
\bigskip
\bigskip

\textbf{5}. Show that if a space $X$ deformation retracts to a point $x\in X$, then for each neighborhood $U$ of $x$ in $X$ there exists a neighborhood $V\subset U$ of $x$ such that the inclusion map $V\hookrightarrow U$ is nullhomotopic.
\medskip

\textbf{Solution}. $\blacksquare$
\newpage

\section{The Fundamental Group}

\subsection{Basic Constructions}

\subsection{Van Kampen's Theorem}

\subsection{Covering Spaces}

\subsection*{Additional Topics}
\addcontentsline{toc}{section}{\protect\numberline{}Additional Topics}

\subsubsection*{1.A. Graphs and Free Groups}
\addcontentsline{toc}{subsection}{\protect\numberline{}1.A. Graphs and Free Groups}

\subsubsection*{1.B. K(G,1) Spaces and Graphs of Groups}
\addcontentsline{toc}{subsection}{\protect\numberline{}1.B. K(G,1) Spaces and Graphs of Groups}

\newpage

\section{Homology}

\subsection{Simplicial and Singular Homology}

\textbf{1}. What familiar space is the quotient $\delta$-complex of a 2 simplex $[v_{0}, v_{1}, v_{2}]$ obtained by identifying the edges $[v_{0}, v_{1}]$ and $[v_{1}, v_{2}]$, preserving the ordering of vertices?
\medskip

\textbf{Solution} The Möbius strip. We draw the same construction as in the reference:

\[\begin{tikzpicture}[scale=0.5]
\coordinate[label=left:$v_{0}$]  (v0) at (0,0);
\coordinate[label=right:$v_{1}$] (v1) at (4,0);
\coordinate[label=above:$v_{2}$] (v2) at (2,3.464);
\coordinate (1/2) at (1,1.732);
\draw (v0) -- (v1) -- (v2) -- cycle;
\begin{scope}[decoration={markings, mark=at position 0.5 with {\arrow{>}}}]
\draw[postaction={decorate}, color=red] (v0) -- node[auto] {} (v1);
\draw[postaction={decorate}] (v0) -- node[auto] {} (v2);
\draw[postaction={decorate}, color=red] (v1) -- node[auto] {} (v2);
\end{scope}
\begin{scope}
\draw[dashed, color=blue] (v1) -- node[auto] {} (1/2);
\end{scope}
\end{tikzpicture}\quad\quad
\begin{tikzpicture}[scale=0.5]
\coordinate[label=left:$v_{0}$]  (v0) at (0,0);
\coordinate[label=right:$v_{1}$] (v1) at (4,0);
\coordinate (v11) at (4.5,0.5);
\coordinate[label=above:$v_{2}$] (v2) at (2.5,3.964);
\coordinate[label=above:$\hat{v}$] (1/2) at (1,1.732);
\coordinate (1/21) at (1.5,2.232);
\draw (v0) -- (v1) -- (1/2) -- cycle;
\draw (v11) -- (v2) -- (1/21) -- cycle;
\begin{scope}[decoration={markings, mark=at position 0.5 with {\arrow{>}}}]
\draw[postaction={decorate}, color=red] (v0) -- node[auto] {} (v1);
\draw[postaction={decorate}] (v0) -- node[auto] {} (1/2);
\draw[postaction={decorate}, color=blue] (v1) -- node[auto] {} (1/2);
\draw[postaction={decorate}, color=red] (v11) -- node[auto] {} (v2);
\draw[postaction={decorate}] (1/21) -- node[auto] {} (v2);
\draw[postaction={decorate}, color=blue] (v11) -- node[auto] {} (1/21);
\end{scope}
\end{tikzpicture}\quad \quad
\begin{tikzpicture}[scale=0.5]
\coordinate[label=left:$v_{0}$]  (v0) at (0,0);
\coordinate[label=right:$v_{1}$] (v1) at (4,0);
\coordinate[label=left:$v_{1}$] (v11) at (0,-0.5);
\coordinate[label=right:$v_{2}$] (v2) at (4,-0.5);
\coordinate[label=above:$\hat{v}$] (1/2) at (2,1.732);
\coordinate[label=below:$\hat{v}$] (1/21) at (2,-2.232);
\draw (v0) -- (v1) -- (1/2) -- cycle;
\draw (v11) -- (v2) -- (1/21) -- cycle;
\begin{scope}[decoration={markings, mark=at position 0.5 with {\arrow{>}}}]
\draw[postaction={decorate}, color=red] (v0) -- node[auto] {} (v1);
\draw[postaction={decorate}] (v0) -- node[auto] {} (1/2);
\draw[postaction={decorate}, color=blue] (v1) -- node[auto] {} (1/2);
\draw[postaction={decorate}, color=red] (v11) -- node[auto] {} (v2);
\draw[postaction={decorate}] (1/21) -- node[auto] {} (v2);
\draw[postaction={decorate}, color=blue] (v11) -- node[auto] {} (1/21);
\end{scope}
\end{tikzpicture}\quad\quad
\begin{tikzpicture}[scale=0.5]
\coordinate[label=left:$v'$]  (v1) at (0,0);
\coordinate[label=right:$\hat{v}$] (v2) at (4,0);
\coordinate[label=above:$v'$] (v3) at (4,4);
\coordinate[label=above:$\hat{v}$] (v4) at (0,4);
\draw (v1) -- (v2) -- (v3) -- (v4) -- cycle;
\begin{scope}[decoration={markings, mark=at position 0.5 with {\arrow{>>}}}]
\draw[postaction={decorate}, color=blue] (v1) -- node[auto] {} (v2);
\draw[postaction={decorate}, color=blue] (v3) -- node[auto] {} (v4);
\end{scope}
\begin{scope}[decoration={markings, mark=at position 0.5 with {\arrow{>}}}]
\draw[postaction={decorate}] (v1) -- node[auto] {} (v4);
\draw[postaction={decorate}] (v2) -- node[auto] {} (v3);
\end{scope}
\end{tikzpicture}\]

The latter being the Möbius strip. $\blacksquare$
\medskip

References: \href{https://riemannianhunger.wordpress.com/solutions-to-algebraic-topology-by-allen-hatcher/hatcher-2-1-1/}{1}.
\bigskip
\bigskip

\textbf{2}. Show that the $\delta$-complex obtained from $\delta^{3}$ by performing the order-preserving edge identifications $[v_{0}, v_{1}]\sim [v_{1}, v_{3}]$ and $[v_{0}, v_{2}]\sim [v_{2}, v_{3}]$ deformation retracts onto a Klein bottle. Also, find the other pairs of identifications of edges that produce $\delta$-complexes deformation retracting onto a torus, a 2-sphere, and $\R P^{2}$.
\medskip

\textbf{Solution}. $\blacksquare$
\bigskip
\bigskip

\textbf{11}. Show that if $A$ is a retract of $X$ then the map $H_{n}(A)\to H_{n}(X)$ induced by the inclusion $A\subset X$ is injective.
\medskip

\textbf{Solution}. $\blacksquare$
\bigskip
\bigskip

\textbf{15}. For an exact sequence $A\to B\to C\to D\to E$ show that $C = 0$ iff the map $A\to B$ is surjective and $D\to E$ is injective. Hence for a pair of spaces $(X, A)$, the inclusion $A\hookrightarrow X$ induces isomorphisms on all homology groups iff $H_{n}(X, A) = 0$ for all $n$.
\medskip

\textbf{Solution}. $\blacksquare$
\bigskip
\bigskip

\textbf{25}. Find an explicit, noninductive formula for the barycentric subdivision operator $S:C_{n}(X)\to C_{n}(X)$.
\medskip

\textbf{Solution}. In general we have the inductive operator taking $\sigma\in C_{n}(X)\to C_{n}(X)$ by

$$B_{p}(\sigma) = b(\sigma)\left(B_{p-1}(\partial\sigma)\right)$$

where $b$ is the barycenter of $\sigma$. For $n = 1$, we have
\begin{align*}
B[v_{0}, v_{1}] &= b([v_{0}, v_{1}])(B\partial [v_{0}, v_{1}]) = b([v_{0}, v_{1}])(B([v_{1}]-[v_{0}]))\\ &= b([v_{0}, v_{1}])([v_{1}]-[v_{0}]) = \left[\frac{v_{0}+v_{1}}{2}, v_{1}\right] - \left[\frac{v_{0}+v_{1}}{2}, v_{0}\right].
\end{align*}

For $n = 2$, we have
$$B[v_{0}, v_{1}, v_{2}] = b([v_{0}, v_{1}, v_{2}])(B\partial [v_{0}, v_{1}, v_{2}]) = b([v_{0}, v_{1}, v_{2}])(B([v_{1}, v_{2}]-[v_{0}, v_{2}] + [v_{0}, v_{1}]))$$
$$= b([v_{0}, v_{1}, v_{2}])\left(\left[\frac{v_{1}+v_{2}}{2}, v_{2}\right] - \left[\frac{v_{1}+v_{2}}{2}, v_{1}\right] - \left[\frac{v_{0}+v_{2}}{2}, v_{2}\right] + \left[\frac{v_{0}+v_{2}}{2}, v_{0}\right] + \left[\frac{v_{0}+v_{1}}{2}, v_{1}\right] - \left[\frac{v_{0}+v_{1}}{2}, v_{0}\right]\right)$$
$$= \left[\frac{v_{0}+v_{1}+v_{2}}{3},\frac{v_{1}+v_{2}}{2}, v_{2}\right] - \cdots + \left[\frac{v_{0}+v_{1}+v_{2}}{3}, \frac{v_{0}+v_{1}}{2}, v_{1}\right] - \left[\frac{v_{0}+v_{1}+v_{2}}{3}, \frac{v_{0}+v_{1}}{2}, v_{0}\right].$$

And now we can see a clear pattern where at each iteration, we add the barycenter of the $n$-th simplex to the image of the operator acting on the $(n-1)$-th simplex. We construct the non-inductive barycenter operator as 

$$B(\sigma_{n}) := \sum_{\pi \in S_{n+1}} \textnormal{sign}(\pi) \left[\frac{\sum_{i=0}^{n} v_{i}}{n+1}, \frac{\sum_{i=0}^{n-1} v^{\pi}_{i}}{n}, \ldots, \frac{\sum_{0}^{1} v^{\pi}_{i}}{1}, v^{\pi}_{0}\right]$$

where $S_{n}$ is the permutation group of $n$ vertices, $\textnormal{sign}(\pi)$ is the orientation of each permutation $\pi$, and where it applies, $v^{\pi}$ means the vertices that belong to the $(n-1)$-simplex of the $\pi$-th permutation. Note that in each element, we are summing over the $i$-th vertex of a permutation, and not the $i$-th index of $\sigma_{n}$. For example, in the last element, $v^{\pi}_{0}$ means the 0-th element of the $\pi$-th permutation, which could mean $v_{0}$, $v_{1}$, $v_{2}$, and so on. It does not strictly mean $v_{0}$. This is exemplified in our example for $n = 2$. $\blacksquare$
\bigskip
\bigskip

\subsection{Computations and Applications}

\subsection{The Formal Viewpoint}

\subsection*{Additional Topics}
\addcontentsline{toc}{section}{\protect\numberline{}Additional Topics}

\subsubsection*{2.A. Homology and Fundamental Group}
\addcontentsline{toc}{subsection}{\protect\numberline{}2.A. Homology and Fundamental Group}

\subsubsection*{2.B. Classical Applications}
\addcontentsline{toc}{subsection}{\protect\numberline{}2.B. Classical Applications}

\subsubsection*{2.C. Simplicial Approximation}
\addcontentsline{toc}{subsection}{\protect\numberline{}2.C. Simplicial Approximation}

\newpage

\section{Cohomology}

\subsection{Cohomology Groups}

\subsection{Cup Product}

\subsection{Poincaré Duality}

\subsection*{Additional Topics}
\addcontentsline{toc}{section}{\protect\numberline{}Additional Topics}

\subsubsection*{3.A. Universal Coefficients for Homology}
\addcontentsline{toc}{subsection}{\protect\numberline{}3.A. Universal Coefficients for Homology}

\subsubsection*{3.B. The General Künneth Formula}
\addcontentsline{toc}{subsection}{\protect\numberline{}3.B. The General Künneth Formula}

\subsubsection*{3.C. H–Spaces and Hopf Algebras}
\addcontentsline{toc}{subsection}{\protect\numberline{}3.C. H–Spaces and Hopf Algebras}

\subsubsection*{3.D. The Cohomology of SO(n)}
\addcontentsline{toc}{subsection}{\protect\numberline{}3.D. The Cohomology of SO(n)}

\subsubsection*{3.E. Bockstein Homomorphisms}
\addcontentsline{toc}{subsection}{\protect\numberline{}3.E. Bockstein Homomorphisms}

\subsubsection*{3.F. Limits and Ext}
\addcontentsline{toc}{subsection}{\protect\numberline{}3.F. Limits and Ext}

\subsubsection*{3.G. Transfer Homomorphisms}
\addcontentsline{toc}{subsection}{\protect\numberline{}3.G. Transfer Homomorphisms}

\subsubsection*{3.H. Local Coefficients}
\addcontentsline{toc}{subsection}{\protect\numberline{}3.H. Local Coefficients}

\newpage

\section{Homotopy Theory}

\subsection{Homotopy Groups}

\subsection{Elementary Methods of Calculation}

\subsection{Connections with Cohomology}

\subsection*{Additional Topics}
\addcontentsline{toc}{section}{\protect\numberline{}Additional Topics}

\subsubsection*{4.A. Basepoints and Homotopy}
\addcontentsline{toc}{subsection}{\protect\numberline{}4.A. Basepoints and Homotopy}

\subsubsection*{4.B. The Hopf Invariant}
\addcontentsline{toc}{subsection}{\protect\numberline{}4.B. The Hopf Invariant}

\subsubsection*{4.C. Minimal Cell Structures}
\addcontentsline{toc}{subsection}{\protect\numberline{}4.C. Minimal Cell Structures}

\subsubsection*{4.D. Cohomology of Fiber Bundles}
\addcontentsline{toc}{subsection}{\protect\numberline{}4.D. Cohomology of Fiber Bundles}

\subsubsection*{4.E. The Brown Representability Theorem}
\addcontentsline{toc}{subsection}{\protect\numberline{}4.E. The Brown Representability Theorem}

\subsubsection*{4.F. Spectra and Homology Theories}
\addcontentsline{toc}{subsection}{\protect\numberline{}4.F. Spectra and Homology Theories}

\subsubsection*{4.G. Gluing Constructions}
\addcontentsline{toc}{subsection}{\protect\numberline{}4.G. Gluing Constructions}

\subsubsection*{4.H. Eckmann-Hilton Duality}
\addcontentsline{toc}{subsection}{\protect\numberline{}4.H. Eckmann-Hilton Duality}

\subsubsection*{4.I. Stable Splittings of Spaces}
\addcontentsline{toc}{subsection}{\protect\numberline{}4.I. Stable Splittings of Spaces}

\subsubsection*{4.J. The Loopspace of a Suspension}
\addcontentsline{toc}{subsection}{\protect\numberline{}4.J. The Loopspace of a Suspension}

\subsubsection*{4.K. The Dold-Thom Theorem}
\addcontentsline{toc}{subsection}{\protect\numberline{}4.K. The Dold-Thom Theorem}

\subsubsection*{4.L. Steenrod Squares and Powers}
\addcontentsline{toc}{subsection}{\protect\numberline{}4.L. Steenrod Squares and Powers}

\end{document}